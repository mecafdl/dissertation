\subsubsection*{DDPG Hyperparameters}

\begin{center}
\begin{tabular}{l|l} 
 \hline
 Parameter & Value\\ 
  \hline
 Optimizer & Adam~\cite{Kingma.2014}   \\ 
 Learning rate (critic and actor)& $3*10^{-4}$  \\ 
 Discount($\gamma$) & $0.99$  \\ 
 Replay buffer size & $10^6$  \\ 
 Number of hidden layers & $3$  \\ 
 Number of units per hidden layer & $256$  \\ 
 Number of samples per mini-batch & $64$  \\ 
 Activation function & ReLU  \\ 
 Output activation function & TanH  \\ 

 Target smoothing coefficient ($\tau$ ) & $0.005$  \\ 
 \hline
 Action dimension size & $6$  \\ 
 Observation dimension size & $18$\\
 Initial exploration phase w/o learning (observations) & $1000$ \\
 
 \hline
\end{tabular}
\end{center}

Actions are limited to $\pm 1$ and scaled by factor $\boldsymbol{s}_a =[5,5,10,1,1,2]$ prior to robot application.
The factor $\boldsymbol{s}_a$ was found empirically during experimentation.

\subsubsection{\skillmodelabbr{} Framework}

The first few parameters are determined by the skill context, and they are chosen by the user.
$T_{o_1}$ denotes the goal pose (the \textit{Container} object) of the insertion skill, $T_{o_2}$ is the initial approach pose (the \textit{Approach} object), and $T_{o,3}$ is the pose of the \textit{Insertable} object.
The region of interest (\roi) defines the environment around $T_{o_1}$ as
\begin{align*}
 \text{\roi}=&[[(-0.02,0.02),(-0.02,0.02),(-0.1,0.1)] \text{m},\\
 &[-0.26,0.26),(-0.26,0.26),(-0.1,0.1)] \text{rad}].
\end{align*}
$f_\text{contact}$ determines when an external force is interpreted as a contact, and it is set to $f_\text{contact}=5$~N.
The maximum external wrench and torque thresholds are given by $\maxwrench=[50 \text{ N}, 30 \text{ Nm}]$, and $\maxtorque=[80,80,80,80,12,12,12]$ Nm.
The maximum execution time is set to $\maxtime=10$ s.
The damping factor is set to critical damping $\boldsymbol{\chi}=[0.7,0.7,0.7,0.7,0.7,0.7]$ and $f_\text{grasp} = 30$~N denotes the gripper grasp force.

Table \ref{tab:materials:parameters} lists the learnable parameters and their respective domains for the insertion skill implementation used for the transfer learning experiment.
In total $24$ continuous parameters are learned.
N/A means that this specific parameter dimension was not learned due to practical reasons, e.g., velocity and acceleration of the gripper during a point-to-point motion where the gripper is not needed.

\begin{table*}[ht!]
    \centering
    \caption{Learnable parameters for the insertion skill}
    \label{tab:materials:parameters}
    \begin{tabular}{|c|c|c|c|}
    \hline
    Parameter & Meaning & Size & Domain \\
    \hline
        $\dot{X}_d$ & Desired velocity & $3\times1$ & $[(0,0.5],[0,1],\text{N/A}]$ \\
        \hline
        $\ddot{X}_d$ & Desired acceleration & $3\times1$ & $[(0,1],[0,4],\text{N/A}]$ \\
        \hline
        $\V{a}$ & Amplitude of wrench oscillation & $5\times1$ & \\
        \hline
        $\V{f}$ & Frequency of wrench oscillation & $5\times1$ & \\
        \hline
        $\Delta X$ & Offset of approach pose & $4\times1$ & \\
        \hline
        $K$ & Stiffness of Cartesian impedance controller & $6\times1$ & $[(0,2000),(0,2000),(0,2000),(0,200),(0,200),(0,200)]$ \\
        \hline
    \end{tabular}
\end{table*}