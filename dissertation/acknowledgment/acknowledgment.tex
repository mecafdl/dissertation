%*****************************************************************************************
%*****************************************************************************************
%*****************************************************************************************
\cleardoublepage\currentpdfbookmark{Acknowledgment}{phd_acknowledgment}
\chapter*{Acknowledgment}

Human intelligence is most noteworthy in our ability to design and use tools.
It is my dream to make robots become intelligent tools, the hammer of tomorrow, an extension of our own mind.
This thesis is my humble attempt to make this dream come true, and thus contribute to solutions for our society's most pressing problems by utilizing robots as the physical embodiment of artificial intelligence.

For the opportunity to complete this work, first and foremost I want to thank my supervisor and mentor Sami Haddadin for his continued support, counsel and feedback during the process of creating this dissertation, as well as for the numerous opportunities to explore various facets of the robotics research landscape.
He gave me the opportunity to do research on the very frontier of robotics technology, and the platform to present my work to the research community, industry, politics and beyond.
The impact that my work has might otherwise not have been possible.

Then, I would like to thank my former colleagues at the Munich Institute of Robotics and Machine Intelligence as as the Institute of Automatic Control at the Gottfried Wilhelm Leibniz Universit\"at Hannover, Anna Adamczyk, Lingyun Chen, Xiao Chen, Diego Hidalgo Carvajal, Alexander Moortgat-Pick, Robin Kirschner, Torsten Lilge, Kim Peper, Edmundo Pozo Fortunic, Dennis Ossadnik, Johannes Ringwald, Moritz Schappler, Jonathan Vorndamme, and Fan Wu for their company, interesting conversations and perspectives, and fruitful collaboration.
I especially want to express my gratitude to Samuel Schneider who supported me with experimental work, Fernando Diaz Ledezma, Dennis Knobbe, Johannes K\"uhn, Erfan Shahriari, Florian Voigt, and Peter So for the many successful team-ups for publications and demonstrators, Rolando Franjga for the invaluable craftsmanship support and pragmatism, and Jan Harder and Regine Hunstein for their commitment and dedication to make MIRMI a place of opportunities that enabled me to write this thesis.
Furthermore, I want to thank Sven Parusel at Franka Emika GmbH for his continued support with the robot hardware I used.

Finally, I would like to thank my parents, my brother and my sister for their support and interest in my work, and most importantly my wife Maria and daughter Kira for their love, support, and patience.