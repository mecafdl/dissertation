The experimental results of the learning architecture have found their way into a continuously developed demonstrator.
It showcased the learning of a difficult peg-in-hole problem and demonstrated the robustness and high performance of the found solution.

\begin{figure}[ht!]
    \includegraphics[width=\textwidth]{figures/architecture/arch_insertion.png}
    \caption{From left to right and top to bottom: A political event at the monastery in Seon \copyright{} CSU, the opening of the Vodafone 5G lab \copyright{} Vodafone, Hannover Messe 2019 \copyright{} TUM, DLD 2019 \copyright{} TUM, AI Council of Bavaria \copyright{} TUM.}
    \label{fig:architecture:insertion}
\end{figure}

It used the learning components of \softwareabbr{} and also served as a testing platform for new developments within the software stack.
It was featured on numerous occasions such as the automatica and Hannover Messe trade fairs, various public and political events, and international conferences.
Figure~\ref{fig:architecture:insertion} shows a collage of impressions.
