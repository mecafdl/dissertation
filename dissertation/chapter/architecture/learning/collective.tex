The collective was the most complex demonstrator coming out of this thesis.
To the best of the author's knowledge it is the first long-standing large-scale robot system.
At first, its standard configuration consisted of $13$ robots, and was later extended to over $70$.
Each robot was equipped with a camera and its own real-time PC that ran \softwareabbr{}.
It was used for numerous smaller experiments in the fields of manipulation learning, planning, telepresence, human-robot interaction and control.
Many of the experiments of this thesis have been done with this setup.

\begin{figure}[ht!]
    \includegraphics[width=\textwidth]{figures/architecture/arch_collective.png}
    \caption{From left to right and top to bottom: The official opening of the Munich Institute of Robotics and Machine Intelligence (MIRMI) \copyright{} TUM, the visit of German chancellor Angela Merkel at MIRMI \copyright{} TUM, the AI.BAY conference 2023 \copyright{} TUM, Falling Walls 2019 \copyright{} TUM.}\label{fig:architecture:learning:collective}
\end{figure}

The collective was often featured on public and political events, trade shows, and internal demonstrations, most notably the opening ceremony of MIRMI, the visit of the German chancellor Angela Merkel, and the Falling Walls conference.
Due to its size and complexity it was only moved once to an external location, namely the official opening of the Munich Institute of Robotics and Machine Intelligence (MIRMI) in the Pinakothek der Moderne.
On other occasions, only one robot was moved and connected to the rest via telepresence.
Figure~\ref{fig:architecture:learning:collective} shows several impressions from various events.

