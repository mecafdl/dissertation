The development of \softwareabbr{} bases on the following design principles:
\begin{itemize}
    \item \emph{Enablement of Tactile Skills}: The concept of tactile skills must be implemented, i.e. the software must be able to 1) process tactile stimuli from a tactile platform; 2) it must implement a tactile controller; 3) it must provide a platform for tactile policies that command twist-wrench pairs; 4) it must be capable of evaluating skill executions.
    \item \emph{Skill Optimization}: The software must be capable of using learning algorithms to optimize skills such that no external software packages have to be integrated.
    \item \emph{Safety}: Safety mechanisms must be built in to ensure a stable and reliable platform for research and demonstrations.
    \item \emph{Maintainability}: The software must be maintainable, i.e. individual components should be replaceable without introducing breaking changes throughout the system.
    \item \emph{Realtime}: All control-related parts of the software must be executable in less than $0.5$~ms to ensure a reliable real-time process when connected to a robot.
\end{itemize}