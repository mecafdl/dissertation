Manipulation learning is enabled by the learning module which is implemented in Python.
The reason for this is the high availability of third-party packages for machine learning in the Python language.
The learning module consists of several connected components, as shown in Fig.~\ref{fig:architecture:mios:learning}.

\begin{figure}[ht!]
\begin{center}
\input{figures/svg/software_learning.pdf_tex}
\caption{Overview and working principle of the learning module}
\label{fig:architecture:mios:learning}
\end{center}
\end{figure}

\subsubsection{Functional Components}

\paragraph{Services}

Services implement the actual machine learning methods and are derived from a service base class which provides a common framework and functionality such as initialization of global parameters, parsing of the problem definition and coordination of the learning process.
Concrete services have to implement method-specific initialization, learning and termination procedures.
Examples for services are the PSP algorithm introduced in Sec. \ref{ch:foundations:learning:algorithms}, CMA-ES and Bayesian optimization.

\paragraph{Interface}

The interface is a network-capable point of communication for the learning module attached to the base service class.
The communication is currently realized by the RPC protocol and allows for starting and stopping learning experiments remotely.

\paragraph{Engine}

The engine is the core mechanism of the learning module.
Services push parameter sets $\params$ to the engine's queue which is processed asynchronously by sending the parameters including default execution settings defined in the problem definition to an available agent.
Note that one service can work with an arbitrary number of agents to sample a given problem.
The engine also takes care of setup, reset and termination routines as well as storing results into a local database.

\paragraph{Knowledge Base}

The knowledge base processes the results that are stored in the database.
For example, it can derive a priori knowledge for transfer learning applications.

\subsubsection{Data Structures}

\paragraph{Problem Definition}

The problem definition is a collection of various data items such as the domain $\domain$, the initial parameters $\params_0$ (if applicable to the used ML method) and cost function design (e.g. weights).
Furthermore, it calculates the final cost function value according to the metrics in Sec. \ref{ch:foundations:learning:metrics}.

\paragraph{Task Result}

A task result is a simple container that holds all available information from a task execution, e.g. cost function value, heuristics and success indicator.




