The \software{} (\softwareabbr{}) has been developed as a software platform for the \platformname{} arm with a focus on real-world manipulation and robot learning.
Over time, it has proven to be an advantageous tool for other research topics as well as demonstrator for various use cases in the context of robot manipulation.
Although it has only been developed and tested for the \platformname{} arm, its code structure is designed in a highly modular way so that it can easily be adapted to other robot platforms by utilizing their respective APIs.
\softwareabbr{} consists of several modules which are coordinated by a core module.
An overview of the most important components is shown in Fig.~\ref{fig:architecture:mios:overview}.
The system is capable of various means of communication such as web socket, UDP or RPC.
It is also compatible with the widely used ROS framework, enabling the use of a range of different third-party packages.
Note that ROS was initially not used as a core component since the modules of MIOS are highly integrated and the node structure of ROS would have brought no benefit.
\softwareabbr{} uses its own mongodb database to save environment data, global parameters and results from learning experiments.
Since the code has been documented over the course of this thesis, it is straight-forward to extend in future development to approach not yet covered manipulation problems.
Additionally, the software has been prepared to run with as little requirements as possible.
Besides the usual requirements of the robot's API \emph{libfranka} itself (e.g. a real-time configured Linux kernel), \softwareabbr{} has only a few dependencies and can even be used in a Docker environment making it almost completely independent from the host system.
It is mostly written in \verb!C++! using the 2017 standard, and Python 3.8.

The most important part is the implementation of the \skillmodelabbr{} model which stretches over several of these modules as explained in Sec. \ref{ch:architecture:mios:modules}.

\begin{figure}[ht!]
\begin{center}
\input{figures/svg/software_overview.pdf_tex}
\caption{Overview of the \softwareabbr{} modules}
\label{fig:architecture:mios:overview}
\end{center}
\end{figure}