

This section introduces the \skillmodel{} (\skillmodelabbr{}) framework \cite{Johannsmeier.2019,Johannsmeier.2023}.
It consists of multiple hierarchical layers.
Each layer models a different aspect of tactile manipulation.
This multi-layer structure descends from a high-level semantic layer down to the hardware system layer that is directly connected to the physical robot platform, which is coupled to the real world (see Fig.~\ref{fig:architecture:ggtwrep:overview}).
$\worldstate \in \worldstatespace$ denotes an element of the world state space $\worldstatespace$ and contains, e.g., the robot poses, external forces, or object positions.
$\percept \in \worldstatespace$ denotes the percept vector, and it contains information received by internal or external sensors (see also Sec.~\ref{fig:foundations:representation:tactile_skill}).
% \subsubsection*{Terminology}

% $\params:=\left[\params_\pi,\params_c \right]$ is the parameter vector, and it represents all externally visible skill parameters. $\params_\pi$, and $\params_c$ are the parameter subsets of the policy layer, and the control layer, respectively.
% $\mtime=\left[\mtime_0,\mtime_1\right]$ denotes the skill time interval with the skill execution start $\mtime_0$ and end $\mtime_1$.

\paragraph{Layers}

The framework layers are described in detail in the sections below.
Each layer receives inputs and additional parameters from the above layer, and provides outputs to the below layer.
They also provide constraints $\constraints$ in terms of the context of the task and the system's limits.
These constraints model the limits of the valid input of the respective layer (e.g., the maximum admissible velocities).
Additionally, the percept processor updates and provides the world state $\worldstate$ to the other components based on the percept vector $\percept$ and internal models.
Figure \ref{fig:architecture:ggtwrep:overview} provides an overview of the \skillmodelabbr{} framework with its different layers.

\begin{figure}[ht!]
    \centering
    The \skillmodelabbr{} model is implemented within \softwareabbr{} as depicted in Fig. \ref{ch:architecture:mios:overview}
The semantic and policy layers are mostly covered by three different classes, i.e. skill class, manipulation primitive class and policy class (abstracted by the skill engine), while the control and hardware layer are taken care of by other modules.
In the following all relevant components are briefly described.
Additionally, in Alg.~\ref{alg:software:esepo:core}-\ref{alg:software:esepo:body} pseudo code for the most important software routines is given.

\paragraph{Core Loop}

The \softwareabbr{} core module contains the base loop for skill execution.
It calls all necessary components in the right order, see Alg.~\ref{alg:software:esepo:core}.

\begin{algorithm}
\caption{Core Loop}\label{alg:software:esepo:core}
\begin{algorithmic}[1]
\Procedure{StartSkill(skill)}{}
\State ActiveSkill $=$ skill
\State $\percept \gets$ UpdatePercept(Body.GetState())
\State ActiveSkill.Initialize($\percept$)
\State ControllerPipeline.Initialize($\percept$)
\EndProcedure
\Procedure{CoreLoop}{}
\While {$\commands.stop = $ false}
\State $\commands \gets $ ActiveSkill.SkillCycle($\percept$)
\State $\torque \gets $ ControllerPipeline.Step($\percept$,$\commands$)
\State $\percept \gets $ Body.Control($\torque$)
\State $\percept \gets$ UpdatePercept($\percept$)
\EndWhile
\EndProcedure
\end{algorithmic}
\end{algorithm}

\paragraph{Skill Class}

The skill class plays a central role for the \skillmodelabbr{} implementation.
It contains the entire semantic layer and the manipulation graph which in turn holds a number of manipulation primitive classes.
The skill class is an abstract base class from which the actual skill implementations inherit.
The base class has a host of different methods to interface with other modules of \softwareabbr{} and to provide various support to the concrete skill instantiations.
All concrete skills have to implement a set of methods to ensure basic functionality.

The skill class runs its cycle in real time as indicated in Alg. \ref{alg:software:esepo:skill}.
In this cycle the different conditions of the semantic layer are checked and acted upon if triggered.
Then, the currently active event conditions are checked and the current primitive is switched if they are triggered.
If no conditions were triggered, the currently active manipulation primitive is called to calculate its next command.

\begin{algorithm}
\caption{Skill}\label{alg:software:esepo:skill}
\begin{algorithmic}[1]
\Procedure{Initialize($\percept$)}{}
\State CycleState $\gets$ Init
\State ActiveMP $\gets$ MPGraph.GetInitialMP($\percept$)
\State Success $\gets$ false
\EndProcedure
\Procedure{SkillCycle($\percept$)}{}
\If {CycleState $=$ Init}
	\If {CheckPreconditions($\percept$) $=$ true}
		\State {CycleState $\gets$ Execution}
	\Else
		\State {$\commands \gets $ ActiveMP.stop()}
		\State {CycleState $\gets$ Execution}
	\EndIf
\EndIf
\If {CycleState $=$ Terminate}
\EndIf
\If {CycleState $=$ Execution}
	\If {CheckErrorConditions($\percept$) $=$ true}
		\State {$\commands \gets $ ActiveMP.stop()}
		\State {CycleState $\gets$ Terminate}
	\EndIf
	\If {CheckSuccessConditions($\percept$) $=$ true}
		\State Success $\gets$ true
	\EndIf
	\If {CheckTerminationConditions($\percept$) $=$ true and Success $=$ true}
		\State {$\commands \gets $ ActiveMP.stop()}
		\State {CycleState $\gets$ Terminate}
	\EndIf
	\If {SwitchMP($\percept$) $=$ true}
		\State ActiveMP $\gets$ MPGraph.GetNextMP()
	\EndIf
	\State $\commands \gets $ ActiveMP.GetCommand($\percept$)
\EndIf
\Return $\commands$
\EndProcedure
\end{algorithmic}
\end{algorithm}

\paragraph{Manipulation Primitive Class}

The manipulation primitive class is a container for multiple policies that can be executed in parallel.
When queried for a command by the skill class, the primitive in turn queries the policies it contains.
It makes sure that the resulting command $\commands$ is valid, such that e.g. no two contradicting pose commands are relayed down the command pipeline.
The most important functionality of the manipulation primitive class is shown in Alg. \ref{alg:software:esepo:primitive}.

\begin{algorithm}
\caption{Manipulation Primitive}\label{alg:software:esepo:primitive}
\begin{algorithmic}[1]
\Procedure{GetCommand($\percept$)}{}
\For {p in Policies}
	$\commands \gets \commands + $ p.GetCommand($\percept$)
\EndFor
\If {CheckCommands($\commands$) $=$ false}
	$\commands.stop = $ true
\EndIf
\Return $\commands$
\EndProcedure
\end{algorithmic}
\end{algorithm}

\paragraph{Policy Class}
The policy class is an abstract base class from which concrete policies can be derived.
The concrete policies have to implement basic functionality such as initialization, command calculation and a termination condition.
Concrete implemented policies are e.g. a point-to-point motion generator, a twist generator, or a Lissajous wrench generator.
In Alg. \ref{alg:software:esepo:policy} the example of a point-to-point motion generator is shown.

\begin{algorithm}
\caption{Policy}\label{alg:software:esepo:policy}
\begin{algorithmic}[1]
\Procedure{Initialize}{}
MotionGenerator.Initialize($T_0,T_g,\dot{\V{X}}_d,\ddot{\V{X}}_d$)
\EndProcedure
\Procedure{GetCommand($\percept$)}{}
MotionGenerator.Step($\percept$)
\State $\commands.\twistd \gets $ MotionGenerator.GetTwist()
\Return $\commands$
\EndProcedure
\end{algorithmic}
\end{algorithm}

\paragraph{Controller Pipeline Class}

The controller pipeline class provides implementations of controllers to bridge the skill commands $\commands$ and the robot hardware.
The pipeline class itself is an abstract base class and multiple concrete pipelines can be derived.
Concrete pipelines have to take care of proper controller execution and context switches, i.e. the proper blending of commands between skill executions.
\softwareabbr{} in its current state provides different pipelines for the various command interface levels of the \platformname{} arm.
In this thesis, a pipeline consisting of a unified force / impedance controller including nullspace control with desired joint torque as output is used, see Alg. \ref{alg:software:esepo:controller} for an example.

\begin{algorithm}
\caption{Controller Pipeline}\label{alg:software:esepo:controller}
\begin{algorithmic}[1]
\Procedure{Initialize($\percept$)}{}
\State CartImpCntr.Initialize($\percept$)
\State NullSpaceCntr.Initialize($\percept$)
\State ForceCntr.Initialize($\percept$)
\EndProcedure
\Procedure{Step($\percept$,$\commands$)}{}
\State CartImpCntr.Step($\percept$,$\commands$)
\State NullSpaceCntr.Step($\percept$,$\commands$)
\State ForceCntr.Step($\percept$,$\commands$)
\State $\torque \gets $ CartImpCntr.Gettorque() $+$ NullSpaceCntr.Gettorque() $+$ ForceCntr.Gettorque()
\Return $\torque$
\EndProcedure
\end{algorithmic}
\end{algorithm}

\paragraph{Robot Body Class}

The robot body class is the low-level bridge between \softwareabbr{} and the FCI of the \platformname{} arm.
It uses the API of \emph{libfranka} to establish connection on various command interface levels, e.g. joint torque, Cartesian twist or joint positions.
Its main functionality is to connect the specific libfranka control loop with the \softwareabbr{} core loop, see Alg \ref{alg:software:esepo:body}.
Furthermore, it provides access to the gripper as well as high-level features such as locking and unlocking the brakes of the robot or searching for its IP address in a local network during the start up phase of \softwareabbr{}.

\begin{algorithm}
\caption{Panda Body}\label{alg:software:esepo:body}
\begin{algorithmic}[1]
\Procedure{Control($\torque$)}{}
\State $\percept \gets $ libfranka.Control($\torque$)
\Return $\percept$
\EndProcedure
\end{algorithmic}
\end{algorithm}
    \caption{Architectural overview of the \skillmodelabbr{} framework}
    \label{fig:architecture:ggtwrep:overview}
\end{figure}

\begin{itemize}
\item The semantic layer decides the currently active high-level state as well as the currently active policy in the policy layer.
Furthermore, it is the point of communication with higher-level components such as learning and planning algorithms.
\item The policy layer holds a set of ordinary differential equations (ODE), which are embedded in a graph structure, that produce tactile commands.
\item The control layer implements a unified impedance / force controller that is fed by tactile commands sent from the policy layer.
It then produces desired motor commands for the system layer. Also, safety mechanisms ensure the system and process constraints.
\item The system layer is the lowest layer, and it sends motor commands from the control layer to the robot hardware.
It also provides the current robot state to the other layers.
\end{itemize}

\paragraph{Objects}
A skill is instantiated by using objects $\objectset$ that define the environment that is relevant to the skill, similar to the definition of manipulation processes introduced in Sec.~\ref{ch:foundations:representation:process}.
Note that all skills also contain the end effector (\textit{End Effector}) as a default object.
It has the handle \sendeffector.

\paragraph{Semantic Layer}
The semantic layer consists of a non-deterministic (non-realtime) component for interface purposes and a deterministic (realtime) component that contains a hierarchical state machine.

The non-deterministic (non-realtime) interface consists of two parts: an execution interface $\semantica$ and a learning interface $\semanticb$, which are represented by the functional mappings
\begin{align*}
&\semantica:\objectset,\params_\pi,\worldstate_0 \rightarrow \req,\effects,\domain, \\
&\semanticb:\objectset,\params_\pi \rightarrow \quality,\worldstate_1.
\end{align*}
$\req \in \worldstatespace$ describes the requirements in terms of the world state for executing the skill, $\effects \in \worldstatespace$ is the effect of the skill when it is applied to a user-defined world state $\worldstate_u$ with objects $\objectset$, and $\domain$ is the domain of valid parameters for every $\param \in \params$.

The deterministic component contains a discrete two-layer state machine that consists of four high-level states; namely, the initial state $\stateinit$, the policy state $\statepolicy$, the error state $\stateerror$, and the final state $\statefinal$.
$\stateinit$ is active in the beginning, $\statefinal$ is active at the end in a nominal case, $\stateerror$ is the end state in case an error occurs, and $\statepolicy$ activates the policy layer.
Three transitions govern the switch behavior at the top level of the state machine.
They directly implement the boundary conditions from the process definition introduced in Sec. \ref{ch:foundations:representation:process}, therefore the explanation of their meaning is omitted here.
However, there are a number of default conditions coming from the robot system:
\begin{itemize}
\item There is a default precondition $\condprei=\{T_\sendeffector \in \roi \}$ which states that the robot has to be within a suitable region of interest (\roi{}).
\item There are three default error conditions $\conderri=\{|\extwrench| > \V{f}_\text{ext,max},T_\sendeffector \notin \roi, t>t_\text{max} \}$ which state that the robot may not exceed the maximum external forces, may not leave the \roi, and may not exhaust a maximum time for skill execution.
\end{itemize}

The policy state $\statepolicy$ contains a state machine layer that is denoted as manipulation graph, and it implements the execution state from the process definition.
The graph may be described by $G(\primitives,\transitionset)$, where $\primitives$ denotes the set of policy approximators (nodes), and $\transitionset$ denotes the set of transitions (edges).
The transitions are conditions that, if true, trigger a switch of the current PA according to the graph structure.

The deterministic component of the semantic layer is represented by the functional mapping
\begin{equation}
f_s: \objectset,\params_\pi,\worldstate \rightarrow \state,\primitive.
\end{equation}

\paragraph{Policy Layer}
The policy layer contains a set of sets (which are denoted policy approximators) of ODEs.
Each policy approximator (PA) implements one process state while maintaining the stated conditions.
The currently active PA $\primitive$ is determined by the semantic layer.
 
A single PA implements a tactile policy $\policy$.

Overall, the policy layer functional mapping is expressed by
\begin{equation}
f_\pi: \state,\primitive,\params_\pi,\worldstate \rightarrow \commands,
\end{equation}
where $\primitive$ and $\state$ denote the currently active PA and high level state as determined by the semantic layer. For $\state \neq \statepolicy$, a default PA
\begin{equation*}
\commands=\left[\begin{array}{c}
     \twistd \\
     \wrenchd
\end{array}\right]
=\left[\begin{array}{c}
     \V{0} \\
     \left[\V{0}^T \quad \mfgrasp\right]^T
\end{array}\right],
\end{equation*}
is activated, where $\mfgrasp$ denotes the current grasp force.

\paragraph{Control Layer}
The control layer receives commands $\commands$ from the policy layer and calculates the desired motor commands $\torqued$.
It uses controllers such as the ones introduced in Sec.~\ref{fig:foundations:representation:tactile_skill}.
Architecturally, the control layer is encoded by the functional mapping
\begin{equation}
f_c: \commands,\params_c,\worldstate \rightarrow \torqued.
\end{equation}
Furthermore, the control layer contains safety mechanisms such as value and rate limitations, collision detection, reflexes, and virtual walls.

\paragraph{System Layer}
The system layer is expressed by the functional mapping
\begin{equation}
f_h:\torqued \rightarrow \percept.
\end{equation}
It defines the control/sensing interface to the robot's hardware system and other devices.
It encapsulates any subsequent hardware-specific control loops and other processes.
It is the implementation of the \emph{Tactile Platform}.

\paragraph{Input Processor}
The input processor holds all of the models for internal and external processes. Examples of internal models are the estimated mass matrix $\hat{\massmatrix}(\q)$, Coriolis forces $\hat{\coriolis}(\q,\dot{\q})$, and gravity vector $\hat{\gravityvector}(\q)$. External models describe the state of environmental elements such as the physical objects handled by the robot.
For example, if the robot were to place an object at a new location, a model of the object would be updated with the new pose.
The processor continuously updates the models by using $\percept$ from the current world state $\worldstate$.
The processor functional mapping is 
\begin{equation}
f_i: \percept \rightarrow \worldstate.
\end{equation}

\paragraph{Task Frame}

The task frame $\taskframe$ defines a coordinate frame relative to the robot's origin frame O so that there exists the transformation ${}^\text{O}T_\taskframe$.
All skill commands $\commands$ are then calculated in the task frame and transformed via this transformation matrix into the frame of origin.