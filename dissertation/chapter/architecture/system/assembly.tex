\softwareabbr{} has been used in a demonstrator that was created based on the collaborative assembly planner presented in \cite{Johannsmeier.2017}.
The theoretical foundations and the experimental analysis of this work can be found in Sec.~\ref{ch:foundations:planning} and Sec.~\ref{ch:experiments:planning}, respectively.
The demonstrator is shown in Fig.~\ref{fig:architecture:system:assembly} on the left.

\begin{figure}[ht!]
    \includegraphics[width=\textwidth]{figures/architecture/arch_assembly.png}
    \caption{(Left) The collaborative assembly station \cite{Johannsmeier.2017}. (Right) The later version of the station at the visit of chancellor Angela Merkel \copyright{} TUM.}
    \label{fig:architecture:system:assembly}
\end{figure}

It was used to showcase several manipulation capabilities such as the compliant interaction with an uncertain environment and safety features such as collision detection.
The demonstrator was e.g. featured at the Hannover Messe 2016.
Later this was extended to also include a phase-based motion generator that is capable of velocity adaptation depending on human co-workers in the vicinity \cite{zardykhan2019collision}.
This case has also been presented when the German Chancellor Angela Merkel visited MIRMI in 2019, see Fig.~\ref{fig:architecture:system:assembly} on the right.