In 2021 a complex robotic art installation called \emph{KI.ROBOTIK.DESIGN} was opened in the Pinakothek der Moderne in Munich \cite{haddadin2021pinakothek}.
It consisted of eight robots equipped with cameras in front of a slowly moving canvas.
Each robot had a number of pens in reach that were used to draw on the canvas.
The drawings were images of daily news sites from around the world or ones created by users via an App.
A ninth robot used a crank to move the canvas.

\begin{figure}[ht!]
    \centering
    \includegraphics[width=\columnwidth]{figures/architecture/pinakothek_setup.png}
    \caption{KI.ROBOTIC.DESIGN in the Pinakothek der Moderne \copyright{} TUM}
    \label{fig:architecture:system:pinakothek:setup}
\end{figure}

\softwareabbr{} was utilized to control the robots.
During the setup of the exhibition new skills such as draw and crank were added, and its inherent telepresence capabilities have been used.
Furthermore, the software stack was extended to handle communication between multiple robots and to provide an interface to container management systems such as Kubernetes.
The art installation is depicted in Fig.~\ref{fig:architecture:system:pinakothek:setup}.

