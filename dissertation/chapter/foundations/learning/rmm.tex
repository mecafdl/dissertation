In a randomly performed demonstration of the tactile skill formalism \skillmodelabbr{} (see Sec.~\ref{ch:architecture:skill}), a significant speedup of roughly $10 \times$ was observed when simply reusing learned parameters from a cylinder insertion in a key insertion task \cite{Haddadin.2018}, leading to learning times of $1$ minute instead of $10$ minutes, see Fig.~\ref{fig:foundations:learning:rmm_pre_exp}.

\begin{figure}
    \centering
    In a randomly performed demonstration of the tactile skill formalism \skillmodelabbr{} (see Sec.~\ref{ch:architecture:skill}), a significant speedup of roughly $10 \times$ was observed when simply reusing learned parameters from a cylinder insertion in a key insertion task \cite{Haddadin.2018}, leading to learning times of $1$ minute instead of $10$ minutes, see Fig.~\ref{fig:foundations:learning:rmm_pre_exp}.

\begin{figure}
    \centering
    In a randomly performed demonstration of the tactile skill formalism \skillmodelabbr{} (see Sec.~\ref{ch:architecture:skill}), a significant speedup of roughly $10 \times$ was observed when simply reusing learned parameters from a cylinder insertion in a key insertion task \cite{Haddadin.2018}, leading to learning times of $1$ minute instead of $10$ minutes, see Fig.~\ref{fig:foundations:learning:rmm_pre_exp}.

\begin{figure}
    \centering
    In a randomly performed demonstration of the tactile skill formalism \skillmodelabbr{} (see Sec.~\ref{ch:architecture:skill}), a significant speedup of roughly $10 \times$ was observed when simply reusing learned parameters from a cylinder insertion in a key insertion task \cite{Haddadin.2018}, leading to learning times of $1$ minute instead of $10$ minutes, see Fig.~\ref{fig:foundations:learning:rmm_pre_exp}.

\begin{figure}
    \centering
    \input{figures/rmm.pdf_tex}
    \caption{Transfer learning observation with a $10$x reduction in the learning time for task B. $\params_0$ denotes the initial parameters drawn from a normal distribution $\mathcal{N}(.)$ with a mean $\mu$ and standard deviation $\sigma$. $\paramsopt_A$ and $\paramsopt_B$ are the optimal parameters for tasks A and B, respectively.}
    \label{fig:foundations:learning:rmm_pre_exp}
\end{figure}

In hindsight, it is hypothesized that the specific task-informed design of the tactile skill learning framework, which heavily depends on well-known principles from human motor control, made this straightforward transfer possible.
This somewhat surprising finding encouraged further investigations into the effects of reusing knowledge from seemingly similar skills, which is also believed to be useful for further theoretical groundwork.

Based on this initial experimental observation, it is believed that a transfer effect for tactile skills is embedded in the developed structure.
This allows us to exploit inherent similarities between skills for faster learning, even without specially designed complex computational transfer mechanisms.
The confirmation of such a behavior could be interpreted as an emerging robot motor memory (RMM) that plays a similar role as the motor memory effect in human motor control.
However, no claim is made here that there is a connection between the two.
Although the similarity between two skills as a whole may not be exactly quantifiable by experimental work, transferability might provide a way to ``sample'' the similarity locally.
In other words, if two tactile skills are similar enough, a learned optimal policy solution for the first skill would speed up the time taken to find an optimal policy solution for the second skill.

To structure and analyze skill similarities and achievable transfer learning, the taxonomy classifications \emph{skill class}, \emph{skill subclass}, and \emph{skill instance} from Sec.~\ref{ch:foundations:representation:taxonomy} are used (see Fig.~\ref{fig:foundations:learning:rmm_hypothesis}).
This classification results in level-$2$, level-$1$, and level-$0$ transfers, respectively.

\begin{figure*}[ht!]
    \centering
    \includegraphics[width=\textwidth]{figures/foundations/rmm_hypothesis.png}
    \caption{Hierarchical three-level classification proposal of tactile manipulation skills}
    \label{fig:foundations:learning:rmm_hypothesis}
\end{figure*}

A transfer learning process is denoted as level-$0$ between different tasks that belong to the same skill instance, level-$1$ from a \textit{skill instance} to another instance from the same \textit{skill sub class}, level-$2$ from a different \textit{skill sub class}, and level-$3$ from another \textit{skill class}.
In this work level-$0$, level-$1$, and level-$2$ transfers are addressed.
A level-$0$ transfer refers to, e.g., a cylinder insertion process in which the task was learned with prior knowledge from its own previous learning.
Level-$1$ indicates that this cylinder was learned with knowledge from a different cylinder.
In a level-$2$ transfer, the cylinder was successfully learned based on the knowledge from, e.g., a key insertion skill.
Based on these definitions and the speedup metric introduced in Sec.~\ref{fig:foundations:learning:metrics} a hypothesis can be stated that on average across all investigated tasks the speedup factor is highest for level-$0$ transfers.
Level-$1$ transfers are second and level-$2$ transfers have the lowest speedup.
This also means that there is a learning effort ratio (LER) that increases with the transfer level.
    \caption{Transfer learning observation with a $10$x reduction in the learning time for task B. $\params_0$ denotes the initial parameters drawn from a normal distribution $\mathcal{N}(.)$ with a mean $\mu$ and standard deviation $\sigma$. $\paramsopt_A$ and $\paramsopt_B$ are the optimal parameters for tasks A and B, respectively.}
    \label{fig:foundations:learning:rmm_pre_exp}
\end{figure}

In hindsight, it is hypothesized that the specific task-informed design of the tactile skill learning framework, which heavily depends on well-known principles from human motor control, made this straightforward transfer possible.
This somewhat surprising finding encouraged further investigations into the effects of reusing knowledge from seemingly similar skills, which is also believed to be useful for further theoretical groundwork.

Based on this initial experimental observation, it is believed that a transfer effect for tactile skills is embedded in the developed structure.
This allows us to exploit inherent similarities between skills for faster learning, even without specially designed complex computational transfer mechanisms.
The confirmation of such a behavior could be interpreted as an emerging robot motor memory (RMM) that plays a similar role as the motor memory effect in human motor control.
However, no claim is made here that there is a connection between the two.
Although the similarity between two skills as a whole may not be exactly quantifiable by experimental work, transferability might provide a way to ``sample'' the similarity locally.
In other words, if two tactile skills are similar enough, a learned optimal policy solution for the first skill would speed up the time taken to find an optimal policy solution for the second skill.

To structure and analyze skill similarities and achievable transfer learning, the taxonomy classifications \emph{skill class}, \emph{skill subclass}, and \emph{skill instance} from Sec.~\ref{ch:foundations:representation:taxonomy} are used (see Fig.~\ref{fig:foundations:learning:rmm_hypothesis}).
This classification results in level-$2$, level-$1$, and level-$0$ transfers, respectively.

\begin{figure*}[ht!]
    \centering
    \includegraphics[width=\textwidth]{figures/foundations/rmm_hypothesis.png}
    \caption{Hierarchical three-level classification proposal of tactile manipulation skills}
    \label{fig:foundations:learning:rmm_hypothesis}
\end{figure*}

A transfer learning process is denoted as level-$0$ between different tasks that belong to the same skill instance, level-$1$ from a \textit{skill instance} to another instance from the same \textit{skill sub class}, level-$2$ from a different \textit{skill sub class}, and level-$3$ from another \textit{skill class}.
In this work level-$0$, level-$1$, and level-$2$ transfers are addressed.
A level-$0$ transfer refers to, e.g., a cylinder insertion process in which the task was learned with prior knowledge from its own previous learning.
Level-$1$ indicates that this cylinder was learned with knowledge from a different cylinder.
In a level-$2$ transfer, the cylinder was successfully learned based on the knowledge from, e.g., a key insertion skill.
Based on these definitions and the speedup metric introduced in Sec.~\ref{fig:foundations:learning:metrics} a hypothesis can be stated that on average across all investigated tasks the speedup factor is highest for level-$0$ transfers.
Level-$1$ transfers are second and level-$2$ transfers have the lowest speedup.
This also means that there is a learning effort ratio (LER) that increases with the transfer level.
    \caption{Transfer learning observation with a $10$x reduction in the learning time for task B. $\params_0$ denotes the initial parameters drawn from a normal distribution $\mathcal{N}(.)$ with a mean $\mu$ and standard deviation $\sigma$. $\paramsopt_A$ and $\paramsopt_B$ are the optimal parameters for tasks A and B, respectively.}
    \label{fig:foundations:learning:rmm_pre_exp}
\end{figure}

In hindsight, it is hypothesized that the specific task-informed design of the tactile skill learning framework, which heavily depends on well-known principles from human motor control, made this straightforward transfer possible.
This somewhat surprising finding encouraged further investigations into the effects of reusing knowledge from seemingly similar skills, which is also believed to be useful for further theoretical groundwork.

Based on this initial experimental observation, it is believed that a transfer effect for tactile skills is embedded in the developed structure.
This allows us to exploit inherent similarities between skills for faster learning, even without specially designed complex computational transfer mechanisms.
The confirmation of such a behavior could be interpreted as an emerging robot motor memory (RMM) that plays a similar role as the motor memory effect in human motor control.
However, no claim is made here that there is a connection between the two.
Although the similarity between two skills as a whole may not be exactly quantifiable by experimental work, transferability might provide a way to ``sample'' the similarity locally.
In other words, if two tactile skills are similar enough, a learned optimal policy solution for the first skill would speed up the time taken to find an optimal policy solution for the second skill.

To structure and analyze skill similarities and achievable transfer learning, the taxonomy classifications \emph{skill class}, \emph{skill subclass}, and \emph{skill instance} from Sec.~\ref{ch:foundations:representation:taxonomy} are used (see Fig.~\ref{fig:foundations:learning:rmm_hypothesis}).
This classification results in level-$2$, level-$1$, and level-$0$ transfers, respectively.

\begin{figure*}[ht!]
    \centering
    \includegraphics[width=\textwidth]{figures/foundations/rmm_hypothesis.png}
    \caption{Hierarchical three-level classification proposal of tactile manipulation skills}
    \label{fig:foundations:learning:rmm_hypothesis}
\end{figure*}

A transfer learning process is denoted as level-$0$ between different tasks that belong to the same skill instance, level-$1$ from a \textit{skill instance} to another instance from the same \textit{skill sub class}, level-$2$ from a different \textit{skill sub class}, and level-$3$ from another \textit{skill class}.
In this work level-$0$, level-$1$, and level-$2$ transfers are addressed.
A level-$0$ transfer refers to, e.g., a cylinder insertion process in which the task was learned with prior knowledge from its own previous learning.
Level-$1$ indicates that this cylinder was learned with knowledge from a different cylinder.
In a level-$2$ transfer, the cylinder was successfully learned based on the knowledge from, e.g., a key insertion skill.
Based on these definitions and the speedup metric introduced in Sec.~\ref{fig:foundations:learning:metrics} a hypothesis can be stated that on average across all investigated tasks the speedup factor is highest for level-$0$ transfers.
Level-$1$ transfers are second and level-$2$ transfers have the lowest speedup.
This also means that there is a learning effort ratio (LER) that increases with the transfer level.
    \caption{Transfer learning observation with a $10$x reduction in the learning time for task B. $\params_0$ denotes the initial parameters drawn from a normal distribution $\mathcal{N}(.)$ with a mean $\mu$ and standard deviation $\sigma$. $\paramsopt_A$ and $\paramsopt_B$ are the optimal parameters for tasks A and B, respectively.}
    \label{fig:foundations:learning:rmm_pre_exp}
\end{figure}

In hindsight, it is hypothesized that the specific task-informed design of the tactile skill learning framework, which heavily depends on well-known principles from human motor control, made this straightforward transfer possible.
This somewhat surprising finding encouraged further investigations into the effects of reusing knowledge from seemingly similar skills, which is also believed to be useful for further theoretical groundwork.

Based on this initial experimental observation, it is believed that a transfer effect for tactile skills is embedded in the developed structure.
This allows us to exploit inherent similarities between skills for faster learning, even without specially designed complex computational transfer mechanisms.
The confirmation of such a behavior could be interpreted as an emerging robot motor memory (RMM) that plays a similar role as the motor memory effect in human motor control.
However, no claim is made here that there is a connection between the two.
Although the similarity between two skills as a whole may not be exactly quantifiable by experimental work, transferability might provide a way to ``sample'' the similarity locally.
In other words, if two tactile skills are similar enough, a learned optimal policy solution for the first skill would speed up the time taken to find an optimal policy solution for the second skill.

To structure and analyze skill similarities and achievable transfer learning, the taxonomy classifications \emph{skill class}, \emph{skill subclass}, and \emph{skill instance} from Sec.~\ref{ch:foundations:representation:taxonomy} are used (see Fig.~\ref{fig:foundations:learning:rmm_hypothesis}).
This classification results in level-$2$, level-$1$, and level-$0$ transfers, respectively.

\begin{figure*}[ht!]
    \centering
    \includegraphics[width=\textwidth]{figures/foundations/rmm_hypothesis.png}
    \caption{Hierarchical three-level classification proposal of tactile manipulation skills}
    \label{fig:foundations:learning:rmm_hypothesis}
\end{figure*}

A transfer learning process is denoted as level-$0$ between different tasks that belong to the same skill instance, level-$1$ from a \textit{skill instance} to another instance from the same \textit{skill sub class}, level-$2$ from a different \textit{skill sub class}, and level-$3$ from another \textit{skill class}.
In this work level-$0$, level-$1$, and level-$2$ transfers are addressed.
A level-$0$ transfer refers to, e.g., a cylinder insertion process in which the task was learned with prior knowledge from its own previous learning.
Level-$1$ indicates that this cylinder was learned with knowledge from a different cylinder.
In a level-$2$ transfer, the cylinder was successfully learned based on the knowledge from, e.g., a key insertion skill.
Based on these definitions and the speedup metric introduced in Sec.~\ref{fig:foundations:learning:metrics} a hypothesis can be stated that on average across all investigated tasks the speedup factor is highest for level-$0$ transfers.
Level-$1$ transfers are second and level-$2$ transfers have the lowest speedup.
This also means that there is a learning effort ratio (LER) that increases with the transfer level.