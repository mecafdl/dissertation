\begin{figure*}[ht!]
    \centering
    \includegraphics[width=\textwidth]{figures/foundations/hierarchy.png}
    \caption{Ranks of the taxonomy of manipulation skills}
    \label{fig:taxonomy:hierarchy}
\end{figure*}

A constructive end-to-end manipulation framework with three distinct structural elements is introduced.
The \emph{manipulation process definition} constitutes the abstraction and the interface that is compatible with the \emph{taxonomy of manipulation skills} (TMS), encoding the skill selection process.
The actual implementation is realized by a \emph{process-compatible tactile skill framework}, which can utilize machine learning while still maintaining stability and guaranteeing safety in its control stack.
With the help of this three-element approach, it is possible for a user or an autonomous planning system to select reliable tactile skills that perform well in industrial contexts.
The skills can be used and optimized without domain-specific robotics or machine learning knowledge.

The process definition is the input to the taxonomy.
It is the interface for process experts such as technicians, robot operators, or shop-floor workers to frame their process knowledge.
Overall, this abstraction is inherently compatible with AI planning systems such as the PDDL-based FastDownward planner \cite{helmert2006fast}, or the ASP-based clingo \cite{gebser2014clingo}.

The developed TMS connects the two domains of process-centric and robot-centric representations.
Specifically, it allows us to map a given process definition to a unique tactile skill model using its underlying classification scheme.
The TMS contains five taxonomic ranks (see Fig.~\ref{fig:taxonomy:hierarchy}) in total; namely, \emph{family}, \emph{domain}, \emph{class}, \emph{subclass}, and \emph{instance}.
Each rank represents a decision layer that reduces the set of possible solution policies for a given input process until a complete solution is determined.
The family rank determines the type of interaction with the environment.
So far, only processes with force interaction were considered in the taxonomy.
The domain rank separates policies on the basis of the interaction forces between the process objects and the environment.
At the class rank, further refinement is done based on well-defined, concrete manipulation steps that lead from the initial state of the process to its final state.
The subclass rank groups the skill solutions according to the distinct geometries of the manipulated objects (e.g., plugs and keys).
Finally, the instance rank contains skill solutions that represent an input process with a concrete physical setup.
Note though that the configuration of the setup (i.e. the object poses) is not defined, meaning that one skill instance can have arbitrarily many tasks, i.e. physical instantiations.

Input processes for the TMS are directly derived from established standards such as the German curricula for trainees in metalworking \cite{Burmester.2020}, electronics \cite{Bumiller.2021}, and mechatronics \cite{Hebel.2020}.
These standards provide the basis for almost any process in today's industry by defining boundary conditions, manipulation steps, requirements, and objectives.
By building on top of standard works, the presented framework is directly compatible with the current needs of industrial companies.
As a first step, the TMS contains processes that range from machine tending (e.g. operating levers and pressing buttons), to assembly (e.g. insertion), or material processing (e.g. bending and cutting).