This chapter introduces the theoretical foundations for the learning architecture in Ch.~\ref{ch:architecture}.
The concept of tactile skills is developed and connected to a formal process definition through a hierarchical taxonomy.
These components are then used in a synthesis pipeline that automatically selects a tactile policy to solve a given input process.
A suitable planning system is described that uses tactile skills in sequences to solve an assembly problem for a collaborative team of humans and robots.
A number of learning algorithms are introduced that are later experimentally evaluated using a number of performance metrics.
Finally, a robot motor memory effect is described that forms the basis for the transfer learning capabilities of the learning architecture.