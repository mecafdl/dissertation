In the following, industrial assembly processes are considered that are typically well understood and planned in advance.
Therefore, the construction plans are known beforehand.
Also, it is assumed that during task execution all necessary tools and parts are fed to the team by suitable mechanisms such as conveyor belts.
In turn, the requirements for autonomy are restricted to specific problems such as being able to execute a set of assembly and manipulation skills, or to communicate with humans and other robots within the context of assembly processes.

A framework for human-robot collaboration is proposed that comprises three different architectural levels: the team-level assembly task planner, the agent-level skill planning, and the skill execution level that is the final decision component above the robot real-time control stack.

The planner at team level performs the task allocation for the agents based on an abstract world model and with the help of suitable cost metrics.
To suitably model the types of assembly processes AND/OR graphs \cite{HomemdeMello.1990} are employed as they implicitly model parallelism.
The team-level planner produces task sequences for every agent via A$^\star$ graph-search, from which it derives task descriptions that are then passed down to the agent's skill execution level.
The agents in turn implement tactile skills as introduced in Sec.~\ref{fig:foundations:representation:tactile_skill}.

The proposed separation in terms of entities (agents, team-level planner) and abstraction leads to an efficient planning framework that can produce optimal nominal task plans to build a desired assembly, while handling failures due to dynamic and uncertain environments in a safe and reliable way on the lower levels of control.

\subsubsection{Preliminaries}

To be able to create (large) assembly plans and design control processes that implement those plans, it is necessary to select a proper representation.
In this chapter such a representation is briefly introduced and reasonable assumptions are made about the structure of the assembly, simplifying the subsequent theoretical analysis.
Note that in this work the generation process of assembly plans is not treated.
Instead, it is assumed that such plans are already at disposal.
A popular algorithm to generate assembly plans is e.g. described in \cite{HomemdeMello.1989}.

In the following, a mechanical assembly is denoted by $\assembly$.
An intuitive representation for a single assembly is the graph of connections $\graph$.
This is obtained by simplifying the graph of contact, which was already used in several works \cite{Fazio.1987,HomemdeMello.1989,Huang.1989}.
The resulting graph of connection contains a node for every part $\assemblypart \in \assemblypartset$, where $\assemblypartset \in \assembly$, and an edge if and only if there is at least one contact between the two respective parts.

The parts $\assemblypart$ are assumed to be solid rigid objects, i.e. their shape remains the same throughout the complete assembly process.
A subassembly $\assemblypartset_s \subset \assemblypartset$ has similar properties as the complete assembly.
If a subassembly consists of more than one part, all the parts are connected, meaning the graph of connections induced by that subassembly is connected.
For the sake of simplicity it is assumed that every part $\assemblypart$ is unique, i.e. they are not interchangeable when they have equal properties (e.g. two screws of the same type).
Furthermore, it is assumed that the geometry and characteristics of the connection of every pair of parts are unique.
Thus, every subassembly implicitly defines the complete geometry of the involved parts as well as their connections.
A subassembly with only one part is considered atomic.

In order to build an assembly an assembly plan is needed, which describes the possibilities of how to assemble a work piece.
In particular, it assumes that all parts are in their designated places and the necessary connections were made already, so that one begins with the correct subassemblies.
The process of executing the assembly plan is called assembly process.
The process of assigning the available workers $\worker \in \workerset$ (for which also the more general term agent is used) to the assembly actions is called task allocation.
A sequence of assembly actions that lead from the initial configuration to the finished product is called an assembly sequence $\assemblysequence$ which, in general, is a partially ordered set of assembly actions $\assemblyaction$.
An assembly action $\assemblyaction$ denotes one step in the assembly process and can be understood as a skill instantiation.

\subsubsection{Assumptions}\label{ch:foundations:planning:assumptions}

Three predicates are introduced to further concretize the scenario which is being investigated.
First, it is assumed that a subassembly $\assemblypartset_s$ is stable, $\text{st}(\assemblypartset_s)$, i.e. its parts remain in contact without applying any external forces and the relative geometry of all parts does not change.
Then, the actions $\assemblyaction$ that are used in the assembly process are classified as geometrically feasible and mechanically feasible:
\begin{itemize}
\item $\text{gf}(\assemblyaction)$ denotes that there exists a collision free path to join the two involved subassemblies.
\item $\text{mf}(\assemblyaction)$ holds if it is possible to permanently establish all necessary connections.
\end{itemize}
\paragraph{Sequentiality}
A plan is sequential if there exists a sequence of assembly actions, involving only one subassembly that can represent the plan.
This means that it is never necessary, though possible, to move two subassemblies at a time.
A plan that holds this property could thus be executed by a robot with a single manipulator.
Whether an assembly can be described by a sequential plan depends on its geometry.
Also, it requires that every subassembly is stable.
\paragraph{Monotonicity}
A plan is called monotone if it contains no action that would break an already established connection or modifies the relative geometry in an already existing subassembly.
Monotonicity is often built into real world assemblies.
\paragraph{Coherency}
A plan is coherent for a graph $\graph$ if every subassembly occurring in the plan corresponds to a connected sub graph of $\graph$.
More formally, there exists a connected graph $\graph$ such that
\begin{itemize}
\item there exists an isomorphism from the set of parts to the set of nodes in $\graph$ and
\item for every subassembly occurring in the plan, the sub graph of G that is induced by that subassembly is connected.
\end{itemize}


\subsubsection{Collaborative Assembly Planning Framework}

\begin{figure}[ht!]
\input{figures/svg/applications_planner_concept.pdf_tex}
\caption{Collaborative assembly planning framework. The top layer depicts the input to the team-level planner, which is an AND/OR graph $Y_\assembly$ resulting from an assembly $\assembly$ (Step I).
The planner then solves the problem of optimal task allocation (Step II) for multiple agents considering a given cost function and constructs a set of assembly sequences (Step III).
The actions from the sequences are then passed to the respective agents (Step IV), which possess corresponding skills $\skill$.
Skills in turn consist of more basic structures, i.e. atomic actions, which eventually map to the real-time-level.}
\label{fig:foundations:planning:concept}
\end{figure}

Figure~\ref{fig:foundations:planning:concept} provides an overview of the collaborative assembly planning framework.
The used notation $\worker \rightarrow \assemblyaction$ denotes a specific allocation of agent $\worker$ to action $\assemblyaction$.
The framework comprises three main levels, which are strictly separated.
However, they communicate bilaterally with each other and share common information.
Although the approach is considered as rather general in principle, the focus here lies on its application to industrial assembly processes. 



The first level, called team level, plans the assembly process from the view of a foreman, i.e. it has information about the possible construction plans, assembly parts, the available agents and other resources.
It knows only actions that can manipulate this world model in an abstract way, i.e. it is domain-specific.
The planning process on team-level solves the problem of task allocation in the assembly process for a team of human and robotic workers.
It should be noted that at this level of abstraction no explicit distinction between human and robotic workers is made, since the team-level planner relays only abstract task descriptions without the need to take into account specific implementations of the necessary skills.
It is possible to distinguish between agents by encoding their respective capabilities into cost functions that are used for planning.
At this point it is important to state how the planner interacts with the agents.
During the search process the planner sends the request to perform a specific action to an agent.
The agent answers with the cost and possibly a further request for an interaction, which in turn is integrated into the planning process.
If an agent is not able to perform a specific action it returns infinite costs.
The planner then uses the received costs from all available agents to determine the optimal task allocation.
Please note further, that the planning process at team-level is offline, see Fig.~\ref{fig:foundations:planning:concept}, hence, fast replanning and refinement methods are out of the planner's scope, despite it might be possible for a manageable assembly complexity.

The second level, called agent level, maps the team-level planning process to the capabilities (skills) of the robot.
In general, the agent level may of course contain also other sophisticated planning systems that plan out sequences of skills to accomplish a given objective.
However, for executing industrially relevant skills that need a high level of expert knowledge, practical experience shows that their automatic planning instead of semi-handcrafting in collaboration with process experts is the significantly less capable approach as of today.
Note that since the agent level is implemented on the robot itself, it is possible to use different robot types.
For this, the systems have to provide the necessary formal semantics to the team level, which obviously requires the systems to be able to execute the same actions in principle.
Another important task of the agent level is the consistent handling of events such as collisions or human induced interruptions of the assembly process.
Although the reaction to the actual event itself is assumed to be handled by the skill level, the agent level either has to deal with the consequences in a way that is consistent with the overall plan on team level, or report a failure in plan execution and request replanning from the team-level planner.

The third level, the skill level, is directly responsible for executing trajectory planning, controllers, etc.
For the experimental analysis of the planning system it is realized by the \skillmodelabbr{} framework (see Sec.~\ref{ch:architecture:mios:ggtwrep}).