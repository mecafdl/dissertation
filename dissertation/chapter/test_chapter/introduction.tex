This chapter discusses the extensive experimental work that verifies the theoretical foundations for tactile skills, skill synthesis, skill learning, and assembly planning.
In Sec.~\ref{ch:experiments:taxonomy} a verification experiment demonstrates the implementation of $28$ different skills that solve a variety of challenging real-world industrial processes.
The results indicate a high degree of robustness and performance as well as performance stability.
It can also be shown that the tactile policies coming from the skill synthesis process are reusable across different skill classes.
In Sec.~\ref{ch:experiments:learning} the learning architecture is experimentally verified.
First, a number of relevant learning algorithms are compared and evaluated with regard to their performance on a difficult insertion problem, which is considered unsolved in robotics.
Second, based on the insights from the first experiments, a novel SVM-based learning algorithm was developed and compared to state-of-the-art deep reinforcement learning approaches on the basis of a key insertion process.
The results show superior learning performance, robustness and achieved manipulation performance.
Third, in an earlier experiment a transfer effect has been observed that accelerates skill learning when using prior knowledge from an apparently similar skill.
In order to investigate this further, a large experimental campaign was conducted that produced several essential insights with regard to similarity among skills such as dependency on geometry and asymmetry in transferability.
Then, Sec.~\ref{ch:experiments:comparison} describes a reference experiment methodology for comparison of human and robot manipulation capabilities. Additionally, the learning performance of the robot was compared with the programming skills of a robot expert.
The results indicate that human-level manipulation performance can already be reached for some skills and that autonomous learning at least achieves similar results to a human expert programmer, albeit slower.
Finally, in Sec.~\ref{ch:experiments:planning} an experiment is shown in which a collaborative assembly planner that is compatible with the developed skill pipeline solves the task allocation problem for a complex assembly, which has also been tested on real-world production cases.
This chapter was written based on \cite{Johannsmeier.2017,Johannsmeier.2019,Voigt.2021,Johannsmeier.2022,Johannsmeier.2023,Johannsmeier.2023b}