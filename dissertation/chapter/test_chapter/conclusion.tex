This chapter describes the extensive experimental work done in this thesis.
Specifically, the taxonomy of manipulation skills is verified with a large number of challenging skills that exhibit high robustness and performance.
This shows that the approach is applicable to a wide range of relevant processes.
The learning architecture is experimentally validated with a number of different learning algorithms and a comparison with state-of-the-art deep learning methods.
The results aid in selecting compatible learning algorithms and show superior results when compared to the state-of-the-art.
Furthermore, a large experimental campaign is described that investigates the architecture’s transfer learning capabilities.
It demonstrates accelerated learning when reusing knowledge to learn new skills and provides insights into the transfer mechanism such as dependency on geometry and asymmetric transferability.
The learning performance as well as achieved manipulation performance were directly compared to human manipulation capabilities and skill programming. The results show that for some skills human performance can already be achieved, while also pointing out the specific gaps that still need to be closed.
Finally, a collaborative assembly problem is solved by an automatic planning system and a human-robot team demonstrating how existing skills can be used automatically to solve even complex tasks.
