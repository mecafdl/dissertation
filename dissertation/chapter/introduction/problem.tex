In robotics research, remarkable progress was achieved in recent years with milestones such as the emergence of lightweight robots \cite{Hirzinger.2002,AlbuSchaffer.2007b,Bischoff.2010,campeau2019kinova,Haddadin.2022}, compliant interaction control \cite{Hogan.1985,AlbuSchaffer.2002,AlbuSchaffer.2007,li2018force} and the establishment of safe physical interaction between robots and humans \cite{Luca.2007,Haddadin.2017b}.
These developments paved the way for robotic manipulation, which is a complex challenge connecting numerous research fields such as robot hardware development, motion and interaction control, interaction policy design, vision, motion and task planning, learning and human-robot interaction.
The advances in these fields brought about systematic approaches for implementing skillful and versatile physical robot manipulation capabilities called manipulation skills.
When considering near-future automated workplaces and the significant performance steps made \cite{Pedersen.2016, Indri.2018}, manipulation skills have gained increasing interest from various application domains. In industrial workshops workers will use intuitive-to-handle robots to do repetitive work \cite{Johannsmeier.2017}, in doctors' offices robot assistants will improve hygiene and reliability of diagnostics \cite{Grischke.2019,seidita2021robots}, and the homes of elderly citizens will be equipped with robot assistants to help manage their daily routine \cite{martinez2018personal,trobinger2021introducing}.
All these robotic applications require sophisticated tactile capabilities to enable a safe, reliable and efficient integration into human-centered processes. However, while workers can rely on systematic curricula specifically developed for their profession (e.g. in Germany \cite{zinggeler2010educational}), there is no such curriculum for robots to learn the capabilities required for their tasks.
A significant challenge for today's researchers and engineers is to transfer skill frameworks to relevant industrial scenarios in order to enable the automation of many tasks that are still manually performed.
There are strict requirements for robot manipulation coming from this context such as safety regulations, a high degree of reliability and robustness, repeatability, reaction to unforeseen events, etc.
These requirements can, at least in principle, be addressed by structured skill frameworks.
However, a typical caveat of structured approaches is the missing compatibility with (machine) learning methods.
As a consequence, skill implementations usually require extensive manual design and robotic expertise.
The currently popular end-to-end approaches to skill modeling \cite{Levine.2015, Gu.2017} focus on this specific problem, when autonomous learning is essentially hard-wired into their architecture.
However, considerable obstacles for the integration of autonomously learned robot skills into industrial processes include the need to collect massive amounts of data, and that these methods may not be able to reliably meet process constraints (in particular safety regulations) without significant intervention in their internal structure.
Furthermore, the required learning time, achievable performance, and amount of computational and energy resources needed are still far from being practical \cite{Thompson.2020}
These limitations must be overcome before large numbers of robots will be able to perform countless manipulation skills in different situations.
In particular, it would be impractical for each robot to learn each skill from scratch.
It is also impractical for human experts to manually program each robot, let alone thousands of devices, which is still today's preferred solution in industrial automation.
The ability to efficiently transfer knowledge from already learned skills to solve new problems would significantly advance robot learning and improve the manipulation performance of robots, allowing them to complete tasks to the high standards that are required in real-world scenarios.
Being able to use prior experience to significantly speed up this learning process would make robots not only more resource-efficient, but also much more flexible to use.
In the end, current approaches are not versatile enough to match the need for highly flexible automation solutions in today's industry.
Structure-based frameworks are often able to implement many different skills but there is so far no way to systematically scale their designs to large numbers of processes.
Similarly, learning-based frameworks lack proper transfer capabilities such that extensive learning for each skill is still required.
An additional problem external to the skill frameworks themselves is the necessity to identify appropriate tactile manipulation skills for each specific physical process (including goals, constraints, subtasks, etc.) without access to robot expert knowledge.
This issue prevents laypeople, e.g. shop-floor workers or technicians, from employing robots to carry out repetitive or demanding tasks, as there generally is a lack of expertise in the implementation of robot manipulation skills.
As a result, companies either rely on manual labor, or have to find and pay rare and expensive experts to set up automation solutions.
As mentioned above, autonomous learning capabilities, at least when used alone, are not yet ready to fill this gap.
