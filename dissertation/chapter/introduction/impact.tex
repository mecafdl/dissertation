Various papers have been published during the course of this thesis, many of which have inspired or supported subsequent work in the research community.
This thesis has opened up several new research paths in industrial manipulation, skill learning, and planning.
The demonstration of the results at international conferences, trade shows and public events has already led to new collaborations, products and investments into further research, and has impacted decision-making in companies and politics.
During this work an extensive software platform for robot manipulation, skill planning and learning has been developed that powers many other research efforts and demonstrators not only at MIRMI but also beyond in various projects and at partners such as Vodafone.
The developed learning capabilities have been shown to be very efficient also for real-world problems, which makes them a feasible candidate for a tech transfer into the industry.
Similarly, is the taxonomy approach a candidate to build an industry-backed curriculum of robot skills. This could even in the near future lead into a new programming paradigm for robots where only process knowledge is required to setup automation solutions.
Current research efforts in geriatronics may directly benefit from the results presented in this work, since the envisioned robot assistant for elderly care requires reliable and performant skills for numerous manipulation problems.
The new field of dentronics has similar requirements in order to realize autonomous assistants in doctor’s offices.


The taxonomy is a first step toward robot curricula that allow robots to automatically acquire and optimize manipulation skills through synthesis and learning.
In this work the focus was on manufacturing skills, but in the long-term the taxonomy can be extended indefinitely, supporting various domains such as healthcare, domestic assistance, logistics, inspection, testing, etc.
This will also drive the spread of robotics into other parts of the society with the potential to combat labor shortage and the effects of the demographic change.
Especially the healthcare sector might benefit from assistants that are equipped with safe and efficient manipulation capabilities.
The presented transfer learning capabilities may lead to shared knowledge bases for robots where solutions to given processes can simply be downloaded and quickly adapted to the problem at hand instead of relearning it from scratch.
This will significantly aid in the democratization of robotics and automation in general.
Given the low energy consumption and computational demands of the presented approach, this work may contribute to the efforts to reduce resource demands while maintaining the current standards of living.
