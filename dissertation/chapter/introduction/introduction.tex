
Imagine your body; do not look at it. Close your eyes and tell me what you see. What is the pose of your body right now? Are you standing, seated, lying? Where are your arms and hands? Do not open your eyes just yet. Now, touch your left knee with your right hand. Did you struggle to find your knee? Chances are you did not. How can we manage to know where our body parts are in space at all times without even looking at them? It is safe to assume that we humans have what is called an internal model of our physical self. That is, we have a representation of our body. Such a representation allows us to accomplish remarkable feats. Have you ever caught an object in mid-air? Certainly, succeeding in that involved not only estimating the trajectory of the object but also the motion that your arm and hand needed to accomplish to reach the catching point. Had your arm dimensions been different from the ones you \textit{have in mind}, or had the accuracy of your knowledge about your arm's pose been insufficient, you would not have caught the object. This internal model of yourself is called the body schema in psychology and neuroscience. It plays a definite role in knowing about the state of our body in space and also in calculating how we move. Consider a different scenario. You are in a pool and have been floating around for quite some time now. By the time you leave the pool, you feel heavier, like you need to put more effort into moving your body. A similar example happens during a workout at the gym. You lift a heavy dumbbell for a number of repetitions, and then, when you are done, it feels like bending your arm takes no effort at all. Your body schema adapted its representation to the situation. The accuracy of this internal model and, certainly, its plasticity play an important role in the kinematic and dynamic control of our bodies. Not only that, but the schema is very much related to our ability to use tools, say a pencil, a hammer, or a golf club. Our body schema is plastic and can temporarily integrate external objects, giving us the capability to manipulate those tools with such dexterity as if they were parts of our own bodies.

Robots have been driving mass production in industries for a good number of decades. They are an essential part of assembly lines in factories. Teams of them take care of tasks demanding high precision. Yet, even as they have been undoubtedly the workforce of automation, they are not as autonomous as we would like them to be. They are designed, constructed, and programmed to be accurate. To really ensure accuracy in their tasks, it is not only the robots that need to move with high precision, but also their environments need to be controlled. Every possibility of interaction needs to be predefined, and disturbances are to be minimized if not completely eliminated. This is precisely the reason why robots in industries normally execute their tasks in confined spaces, having little or no interaction at all with the external environments, let alone with humans. However, in recent decades, there has been a shift. Technology has evolved enough so that robots are now leaving these enclosed and protected environments and are aiming at increased interaction with their surroundings, other robots, and people. This advanced interaction comes at a price: uncertainty. The world cannot be modeled; every single potential interaction scenario cannot be anticipated and controlled. Robots, if they are to succeed in the human world, need to improvise strategies, adapt to the situation, and overcome constant, ever-changing challenges. But robots are not (at least until now) adaptable; the ways in which they are modeled are fixed and rarely subject to change. Their control paradigms also do not account for changes. Naturally, their very own bodies are not supposed to change. But change is there, a constant factor. Wear and tear is a typical phenomenon. A localized malfunction is an event that, although undesired, could occur, and a robot should be able to handle it. Perhaps the task that it needs to execute has parameters slightly different from what was anticipated; tools could be different, and the environment could be different. To put it bluntly, modern robots need to have plasticity in their models, just as humans do.

Let us go back to our imagination. Say you are left-handed. Unfortunately, you had a light accident and broke your left arm. Nothing serious, but you will not be using that arm for a couple of weeks. You can adapt your internal model to account for this situation and will probably become dexterous with that good right arm. If a humanoid robot is programmed to accomplish tasks with both arms and one ceases to operate for some reason, it is unlikely that I will carry on with the task. At this point, the engineers responsible for this robot may determine that the robot can be repurposed while the arm is repaired. They adapt the model of the robot, tune some parameters, and off the robot goes to work on its new tasks. But this is not the level of autonomy we imagine when we see a future in which robots coexist with us and assist us in a myriad of things. Once again, robots need to adapt. And the very first aspect that needs to be adapted is their own body. For that, robots need to have the capability to understand and construct a model of their bodies. This brings back the discussion to the idea that a robot can certainly benefit from the understanding of its physical self, and for that, it would need to leverage information coming from its on-board sensors and fuse them with ground truths to make sense of its body and the world around it. But how?


% ===================================================================================================
%                                                 |                                                 |
% -------------------------------------------- SECTION ---------------------------------------------|
%                                                 |                                                 |
% ===================================================================================================
\section{Motivation and vision}
\paragraph*{Empowering Robots to Learn and Control their Bodies} As the integration of robots into various facets of human life becomes more pronounced, there arises a crucial imperative for these machines to proactively participate in the exploration and development of models for their own bodies. At the core of this autonomous self-discovery is the development of an awareness of the physical self by developing and maintaining a body schema. This self-awareness represents a pivotal step towards endowing robots with a profound understanding of their own embodiment and becomes the foundation for the integration of sensory information and motor control. The robot body schema, akin to the internal representation of the human body in the brain, becomes a dynamic and evolving map that serves as a reference point for the robot's interactions with its surroundings. The establishment of a body schema contributes significantly to the enhancement of robot motor control. Robots, equipped with a coherent internal model of their bodies, can execute movements with a heightened level of precision and coordination. The fusion of sensory information and motor control within the body schema lays the scaffolding for efficient learning. Additionally, autonomous self-discovery is not a static process. It involves ongoing refinement and adaptation. As robots encounter diverse environments and engage in various tasks, their body schema adapts and expands, allowing for an extended understanding of their physical capabilities. This adaptability is crucial for robots to navigate the intricacies of real-world scenarios, adjusting their responses based on the context in which they operate. The significance of this autonomous self-discovery transcends the technical aspects of robotics; it resonates with the broader narrative of robots becoming integrated and adaptive participants in human-centric environments. By actively engaging in the exploration of their own bodies, robots pave the way for a future where they seamlessly navigate, interact, and adapt in tandem with human activities. This not only enhances their functional capabilities but also fosters a harmonious integration of robots into diverse and dynamic human-centric spaces.

\paragraph*{Learning and the body schema} The seamless integration of the body schema is indispensable for the evolution of robots, forming the foundational underpinning upon which multifaceted capabilities are constructed. This integration spans across various domains, encompassing learning for physical awareness, motor control, coordination, and interaction. Learning, a hallmark of intelligent systems, coexists in a dual relationship with the body schema: learning not only contributes to the development of the body schema but, reciprocally, the body schema enhances learning. Primarily, it plays a pivotal role in detecting the structure within the myriad afferent and efferent signals of the robot's sensorimotor system, facilitating the construction of the body schema. Simultaneously, the incorporation of the body schema into learning frameworks allows robots to explore their sensorimotor maps and develop models of their morphology. This understanding becomes a scaffold for acquiring new skills, refining existing ones, and assimilating knowledge gained from interactions with the environment. The integration of the body schema thus catalyzes a learning process that is not only adaptive but also inherently tied to the robot's physical embodiment. Moving to motor control, another pivotal aspect of intelligent systems, there exists an intricate link to learning the body schema. Internal body representations that can adapt through learning mechanisms contribute to the development of superior forward and inverse models, ultimately refining control precision. Lastly, a seamlessly integrated body schema empowers robots to learn diverse tasks, surpassing the limitations of rigid, pre-programmed functionalities. Ultimately, a plastic body representation provides versatility crucial in dynamic and unpredictable settings, where the cohesive ability to learn, control movements, coordinate actions, and perceive space allows robots to thrive in a myriad of scenarios.

\paragraph*{Enhancing locomotion, manipulation, and adaptability.} Taking inspiration from the nuanced abilities of humans, the vision for future robots is set to take advantage of the enhanced bodily awareness that the integration of a body schema will bring about, thereby improving adaptability and interaction in diverse situations. The development and maintenance of the robot body schema will unlock a spectrum of capabilities that enables precise and coordinated movements, fostering in particular advanced locomotion, motion planning, and intricate manipulation. Bodily awareness is expected to revolutionize the locomotive prowess of robots. Future machines, informed by this internal map of their physical structure, will navigate environments with an unprecedented level of sophistication. This extends beyond basic movement, enabling robots to traverse complex terrains, negotiate obstacles, and adapt seamlessly to changes in their surroundings. Motion planning, a core element of robotics, will harness the robot body schema as a dynamic blueprint, supporting not only precise and efficient movements but also the ability to determine near-optimal trajectories in real-time. The result is a more adaptive and resourceful approach to navigating intricate spaces and executing tasks with heightened precision. The mastery of a body schema extends to manipulation with profound implications. Future robots, leveraging this internal map, will exhibit a level of dexterity and precision in manipulating objects that mirrors the intricacies of human hand-eye coordination. This enhanced capability will represent a breakthrough in applications requiring delicate and precise interactions, from handling diverse items to executing complex manufacturing tasks. Moreover, coordination and collaboration with other robots or humans becomes more refined through a well-integrated body schema, as it allows robots to anticipate and adapt their interactions with other agents. This anticipatory and adaptive capability is fundamental for fostering safe, effective, and harmonious interactions.

\paragraph*{Constant self-monitoring for autonomy} Continuous self-monitoring signifies a fundamental imperative for future robotic systems. It is achieved through the amalgamation of internal models and the uninterrupted stream of sensorimotor signals. This seamless fusion enables a dynamic and real-time understanding of the robot's own state, creating a powerful feedback loop that is integral to the system's autonomy. Perpetual monitoring sets in motion successive phases of error detection and correction. This iterative process ensures that the robot is not only aware of its physical state but also capable of recognizing and rectifying discrepancies between intended actions and actual outcomes. The ability to identify errors in real-time positions future robots on a trajectory towards enhanced reliability and precision in its interactions. The profound impact of this ongoing adaptation is most evident in the rapid formulation and execution of contingency motion strategies. Armed with an enriched spatial awareness and a continuously evolving set of internal models, the robot becomes adept at anticipating and responding to unforeseen challenges. In dynamic and unpredictable environments, the ability to swiftly devise and implement contingency plans allows the robot to navigate complex scenarios with agility and efficiency. This advanced interaction capability with the environment is a hallmark of the paradigm of continuous self-monitoring. The robot not only perceives its surroundings in real-time but also possesses the foresight to proactively engage with its environment. This goes beyond mere reactionary responses; it encapsulates a proactive and intelligent engagement that significantly elevates the robot's efficacy in accomplishing tasks and navigating diverse scenarios.

\paragraph{Onboard sensing for self-sufficiency} Achieving true autonomy requires robots to evolve into self-sufficient entities capable of independent learning, calibration, monitoring, and adaptation of their body representation. This transformation is predicated on the exclusive reliance on onboard sensing modalities, a fundamental transition that empowers robots with heightened versatility and adaptability. At the core of this self-sufficiency lie two fundamental sensory modalities: somatosensation, encompassing proprioception and touch, and vision. These modalities collectively provide the robot understanding of its own body and the surrounding environment. Proprioception provides the robot awareness of its own body in space. The sense of touch complements the understanding of the body and allows the robot to distinguish itself from its immediate environment. Vision, another cornerstone modality, extends the robot's perception beyond immediate physical contact. By abstaining from off-board sensing devices, robots liberate themselves from external dependencies and enhance their self-sufficiency. Leveraging onboard sensing modalities empowers robots to dynamically respond to changes in their surroundings in real-time. Whether it's navigating through a cluttered environment, adjusting movements for safety, or adapting to unforeseen obstacles, the reliance on somatosensation and vision will enable future robot to operate with a level of autonomy and adaptability previously unseen.

\paragraph*{Safety- and energy-awareness} Comprehending their own body structure empowers robots to engage in more effective and nuanced interactions with both their robotic and human counterparts. The body schema serves as a predictive tool, endowing robots with the foresight to anticipate potential challenges. This heightened understanding facilitates dynamic adjustments in their movements, prioritizing safety and fostering efficiency in diverse collaborative scenarios. When interacting with other robots, this capability enables seamless coordination, averting collisions or disruptions during joint tasks. Likewise, in human-robot interactions, the ability to adapt movements ensures a safer environment, mitigating the risk of accidental impacts or collisions. Moreover, the comprehension of their own body structure provides robots with a unique advantage in optimizing energy consumption. By adapting their motions based on the inherent physical properties of their structures, robots can execute tasks with greater efficiency. This optimization is particularly crucial in mobile robotics, where energy conservation is paramount given the limited resources. The dual capability of enhancing safety in interactions and contributing to energy-aware robotics underscores the significance of robots understanding their own body structure in fostering a more efficient, collaborative, and environmentally conscious robotic landscape.


% ===================================================================================================
%                                                 |                                                 |
% -------------------------------------------- SECTION ---------------------------------------------|
%                                                 |                                                 |
% ===================================================================================================
\section{Problem Statement}
Navigating the complexities of learning a robot's physical attributes in the field of robotics reveals intricate challenges that demand a closer look. Traditional calibration routines, tailored for known kinematic structures and offline system identification methods, face limitations concerning generalization capabilities, sample efficiency, and the reliance on a predetermined mechanical topology. This is particularly evident in the context of global and local machine learning frameworks for physical systems, where the exclusion of structural knowledge often results in operational inefficiencies. 

Compounding these challenges is the substantial reliance on off-robot measurement devices, such as vision and motion-capturing systems, during calibration and identification processes. Despite the wealth of sensor signals from modern robots, determining the minimum set required for constructing a body model based solely on robot sensing remains a persistent and unresolved challenge. This dependence on external measurements not only introduces limitations but also raises questions about the practicality and applicability of learned models in real-world scenarios.

Venturing into alternative learning methods, such as neural networks, introduces a unique set of challenges. These approaches frequently lack crucial information about the robot's body structure, necessitating vast amounts of data for effective learning. The design of neural networks further compounds the issue, as it becomes an expert-driven task requiring meticulous determination of topology. Often, these approaches suffer from generalization limitations, confining their learning capabilities to specific input-output regions

Examining the research landscape reveals substantial gaps, including an unclear understanding of how object handling extends the robotic body schema. Furthermore, limited exploration into the mechanical arrangement of joints and links, known as the mechanical topology, underscores untapped potential in understanding the intricate details of robotic physical structure. These research gaps collectively contribute to the overarching challenge of lacking a unifying scheme that seamlessly integrates all learning stages, hindering the development of a fully characterized robotic body schema based solely on knowledge about sensorimotor signals.

To address these multifaceted challenges effectively, there is a compelling need to bridge the existing gaps and explore innovative learning approaches. This includes the development of methods capable of incorporating structural knowledge, reducing reliance on external measurements, and enhancing the generalization capabilities of machine learning frameworks in the realm of robotics. Moreover, the establishment of a comprehensive unifying scheme stands as a pivotal step toward advancing the field and achieving a fully characterized robotic body schema.

% ===================================================================================================
%                                                 |                                                 |
% -------------------------------------------- SECTION ---------------------------------------------|
%                                                 |                                                 |
% ===================================================================================================
\section{Research Questions and Contribution}
%The set of fundamental research questions that are addressed in this thesis aim to provide convincing answers to help push the existing boundaries of robotics research and generate solutions to several of the aforementioned challenges:
\begin{itemize}
    \item Q1: How can robot expert knowledge be systematically encoded such that non-experts can use robots to automate complex manipulation processes?
    \item Q2: For this, it is essential to understand how tactile manipulation skills can be encoded such that control, policy, learning, and planning are unified.
    \item Q3: To generate breakthroughs in the robot learning problem, it needs to be understood how complex robot manipulation skills can be learned in a short amount of time and with low energy consumption and computational demands such that the solution is robust and of high performance.
    \item Q4: Finally, this thesis aims to give answers to the question of whether systematic similarities between skills can be exploited by new ways of transfer learning so skill learning can be scaled up to large numbers of skills without running into the curse of dimensionality.
\end{itemize}

This thesis has made three core contributions that provide answers to above research questions.
The contributions can be clustered into three major groups, namely tactile skill modeling, automatic skill synthesis, and autonomous skill learning.
In Fig.~\ref{fig:introduction:title} they are summarized in an overview.

\begin{figure}[ht!]
    \centering
    \input{figures/title_figure.pdf_tex}
    \caption{The process descriptions that originate e.g. from human technicians are used to synthesize tactile skills based on a taxonomy. The skills are formulated as a learning problem and then optimized by an autonomous learning architecture. Finally, the optimized skills are used in an assembly planning system that solves problems provided by the technician.}
    \label{fig:introduction:title}
\end{figure}

\paragraph{Contribution 1: Tactile Skill Modeling}

The theoretical foundations for tactile skills are provided, consisting of tactile platform, tactile controller, tactile policy, and performance evaluator.
The \skillmodel{} (\skillmodelabbr{}) framework is introduced as an algorithmic means of modeling tactile skills.
It connects task goals with interaction control through a process-informed multi-layered structure and forms the basis for most experiments in this thesis.
Furthermore, a versatile software stack \softwareabbr{} has been developed that implements the theoretical foundations.

\paragraph{Contribution 2: Skill Synthesis Pipeline}

A novel process-informed skill taxonomy is introduced that systematically connects manufacturing processes and tactile skills. It encodes robot expert knowledge and connects to learning algorithms so that laypeople such as technicians and shop-floor workers will be enabled to set up automation solutions without robotics knowledge.
A synthesis pipeline selects a tactile policy based on formal process descriptions.
The \skillmodelabbr{} framework then takes the selected policy and integrates it with the boundary conditions (e.g. success and error conditions) from the process description into a tactile skill model.
In further validation experiments, a meaningful number of skills has been implemented and optimized, demonstrating the desired high robustness and performance also for challenging real-world manufacturing processes.
A collaborative assembly planner is introduced that finds an optimal plan to solve an assembly with a team of humans and robots based on validated tactile skills.

\paragraph{Contribution 3: Skill Learning}

The skill model generated from the synthesis pipeline is formulated as a parameter learning problem, which can then be solved by a state-of-the-art compatible learning algorithm.
Experiments demonstrate a very short learning time even for very difficult processes, e.g., such as insertion. 
During these experiments a transfer learning effect was observed and was investigated through further large-scale experiments. The results indicate strong transfer capabilities of the \skillmodelabbr{} framework which further improves the versatility of the overall end-to-end approach.
The learning and achieved manipulation performance of the skills were directly compared to human capabilities in a reference experiment setup. The results indicate comparable performance for a subset of the implemented skills, and provide insights on the still-existing gaps.
\newline
\newline
With these fundamental contributions and to the best of the author's knowledge an unprecedented level of sophistication, performance and resource efficiency in terms of energy and computation has been achieved in robot learning. This constitutes a major step forward in making robots autonomous and learning-enabled as well as able to systematically leverage existing and well established process knowledge. 



% ===================================================================================================
%                                                 |                                                 |
% -------------------------------------------- SECTION ---------------------------------------------|
%                                                 |                                                 |
% ===================================================================================================
\section{State of the Art}
%In this section the verification experiment for the taxonomy of manipulation skills is presented.
First, the general hardware setup and the specific process setups are described. Then the verification process is outlined and the results are shown.
Finally, the results are discussed.

\subsection{Experimental Setup}
The taxonomy verification experiments are based on a common hardware base setup that consists of the following components.
\begin{itemize}
\item A \platformname{} arm \cite{Haddadin.2022}: This is a 7-DOF manipulator with link-side joint torque sensors and a $1$ kHz torque-level real-time interface, which allows us to directly connect the \skillmodelabbr{} framework to the system hardware, i.e., the real-time interface FCI \cite{Haddadin.2022}.
\item A \platformname{} hand: A standard two-fingered gripper that is sufficient for a wide range of tasks.
\item Intel NUC: A small PC that uses an Intel i7 CPU, 16 GB RAM, and an SSD. \footnote{The used learning approaches do not require GPU acceleration or distributed computing clusters.}
\item Software: \softwareabbr{} is used.
\end{itemize}

Input processes for the taxonomy of manipulation skills (TMS) are directly derived from established standards such as the German curricula for trainees in metalworking \cite{Burmester.2020}, electronics \cite{Bumiller.2021}, and mechatronics \cite{Hebel.2020}.
These standards provide the basis for almost any process in today's industry by defining boundary conditions, manipulation steps, requirements and objectives.
By building on top of standard works the robot manipulation framework is then directly compatible with the current needs of industrial companies.
As a first step, the TMS contains processes that range from the domain of machine tending (such as lever operation and button pressing), to assembly (such as insertion), or material processing (such as bending and cutting).
To illustrate the power of the framework, $28$ real-world manipulation skills were implemented within the \skillmodelabbr{} framework as described in Table \ref{tab:experiments:taxonomy:setup}. 


\subsection{Verification Process}
For the validation experiment, each skill model was executed $50 \times$ on the same setup.
A single trial involves executing a particular skill model until it terminates.
When appropriate, artificial errors were used to offset the skill's goal poses in the validation experiment to simulate a more realistic process environment.
For example, in typical industrial environments, moving parts of heavy machines cause process disturbances that impact the robot's precision.
The process-specific experiment setups are depicted in Fig. \ref{fig:experiments:taxonomy:taxonomy}.
Considering the validation experiment, as well as the optimization experiments (both autonomous learning and manual tuning), roughly $6000$ trials were run, i.e., executions, for a single skill.
Taking into account the optimization times and setup times (i.e., physically adjusting the robot's environment for the next experiment), the entire experimental work took about one month to complete.

\begin{figure}[ht!]
    \centering
    \includegraphics[width=\textwidth]{figures/experiments/taxonomy.png}
    \caption{Taxonomy of manipulation skills, experimentally validated skills are shown with the used setup. For clarity, the taxonomy ranks are indicated as family (F), domain (D), class (C), and subclass (S). Instances are omitted since they are represented by the pictures.}
    \label{fig:experiments:taxonomy:taxonomy}
\end{figure}

\renewcommand{\arraystretch}{0.95}
\begin{table*}[!ht]
\centering
\caption{Setup descriptions}
\label{tab:experiments:taxonomy:setup}
\resizebox{\textwidth}{!}{%
\begin{tabular}{|p{1.5cm}|p{13cm}|c|}
\hline 
Insert & A cylinder ($\err{x,y,z}{0.003}$) with $30$ mm diameter and very low tolerances ($<0.1$ mm), a household key, and an Ethernet plug (both $\err{x,y,z}{0.001}$). & $\policyi{27}$
\\
\hline
Tip & A mechanical enter key and two different spring-loaded buttons.
For the enter key $\transition_2$ was automatically triggered when the key was hit properly.
$\err{x,y,z}{0.001}$ for all three cases.& $\policyi{23}$
\\
\hline
Drag & A box filled with objects, resulting in roughly $2.1$ Kg of weight, was dragged over three different surfaces, i.e., wood, cloth and foil.& $\policyi{3}$
\\
\hline
Slide & A common computer mouse and three different surfaces, i.e., wood, cloth and foil.
The robot had to maintain a contact force of $15$ N.& $\policyi{6}$
\\
%\hline
%Shove & A box filled with objects, resulting in roughly $2.1$ Kg of weight, was shoved over three different surfaces, i.e. wood, cloth and foil.\\
\hline
Press Mechanism & A pedal ($\err{x,y,z}{0.01}$), a user stop, and a flip switch (both $\err{x,y,z}{0.005}$).
The pedal  must be pressed for $1$ s, the other two buttons have no minimum press time.
The user stop required significantly more force to be pressed down than the other two button variations.
The press button skill was only optimized for contact torques, since the execution time is mostly determined by the press time.& $\policyi{33}$
\\
\hline
Extract & A cylinder ($\err{x,y,z}{0.003}$) with $30$ mm diameter and very low tolerances ($<0.1$ mm), a household key, and an Ethernet plug (both $\err{x,y,z}{0.001}$).& $\policyi{34}$
\\
\hline
Cut & A cutter knife on a carton surface, a cloth surface and a foil surface.
Success was confirmed manually and by visual inspection by a human experimenter depending on whether the surface has been cut properly.& $\policyi{26}$
\\
\hline
Grab & An object (HDMI switch in this case) on a table with $\err{x,y,z}{0.005}$. Note that, the grab skill is very easy to test since it is a position-based task for the most part with little physical interaction.
The grab skill was only optimized for execution time due to the minimal amount of physical interaction.& $\policyi{35}$
\\
\hline
Place & An object (HDMI switch in this case) grasped by the robot to be placed on a flat table  with $\err{x,y,z}{0.005}$.
The place skill was only optimized for execution time due to the minimal amount of physical interaction.& $\policyi{36}$
\\
\hline
Swipe & A stylus on a tablet with $\err{x,y,z}{0.005}$.
The success of the skill was determined visually by the human experimenter depending on a successful swipe operation on the tablet.& $\policyi{26}$
\\
\hline
Bend & Two wooden plates connected by heavy cables which allow for a reset without too much wear on the setup ($\err{y}{0.005}$).& $\policyi{3}$
\\
\hline
Turn Mechanism & A key inserted into a lock ($\err{x,y,z}{0.005}$).& $\policyi{2}$
\\
\hline
Slide Off & A battery casing with a slideable lid ($\err{x,y,z}{0.005}$).
The success of the skill was determined visually by the human experimenter.& $\policyi{4}$
\\
\hline
Move Mechanism & A common lever ($\err{y}{0.005}$).& $\policyi{1}$
\\
\hline
\end{tabular}
}
\end{table*}


\subsection{Results}
Overall, high performance levels could be achieved in terms of the two cost functions' minimum \emph{execution time} and \emph{contact moments}.
Table \ref{tab:experiments:taxonomy:results} summarizes the achieved \emph{robustness} against the \emph{goal pose randomization} and the \emph{average and standard deviation of these metrics}.
The skills were either autonomously learned or domain expert-tuned.

\renewcommand{\arraystretch}{0.9}
\begin{table*}[ht!]
\centering
\caption{Experimental results for all skills}
\label{tab:experiments:taxonomy:results}
\resizebox{\textwidth}{!}{%
\begin{tabular}{|c|c|c|c|c|c|}
\hline 
\multirow{2}*{Skill} & \multirow{2}*{Task} &  \multicolumn{2}{|l|}{Execution Time} &  \multicolumn{2}{|l|}{Contact Torques} \\ 
\cline{3-6}
 &  & Robustness & Value & Robustness & Value \\ 
\midrule 
\multirow{2}*{Insertion} & Cylinder & $94$ \%  & $1.76 \pm 0.36$ s& $92$ \% & $3.21 \pm 0.35$ Nm \\ 
\cline{2-6}
& Key & $100$ \%  & $1.1 \pm 0.18$ s& $98$ \% & $2.97 \pm 0.244$ Nm \\ 
\cline{2-6}
 & Ethernet Plug & $100$ \%  & $1.04 \pm 0.19$ s& $100$ \% & $2 \pm 0.55$ Nm \\ 
\midrule
\multirow{2}*{Extraction} & Cylinder & $100$ \%  & $0.35 \pm 0.05$ s& $100$ \% & $6.03 \pm 0.123$ Nm \\ 
\cline{2-6}
&Key & $100$ \%  & $0.31 \pm 0.02$ s& $100$ \% & $5.46 \pm 1.277$ Nm \\ 
\cline{2-6}
&Ethernet Plug & $100$ \%  & $0.23 \pm 0.001$ s& $100$ \% & $2.29 \pm 0.023$ Nm \\ 
\midrule 
\multirow{2}*{Press Mechanism} & Pedal & $100$ \%  & N/A & $100$ \% & $4.12 \pm 0.223$ Nm \\ 
\cline{2-6}
&Flip Switch & $100$ \%  & N/A & $100$ \% & $1.6 \pm 0.083$ Nm \\ 
\cline{2-6}
& User Stop & $100$ \%  & N/A & $98$ \% & $3.39 \pm 0.17$ Nm \\  
\midrule
\multirow{2}*{Tip} & Enter Key & $100$ \% & $0.82 \pm 0.005$ s & $100$ \% & $0.59 \pm 0.013$ Nm \\ 
\cline{2-6}
&Red Button & $100$ \% & $0.69 \pm 0.034$ s & $100$ \% & $1.41 \pm 0.137$ Nm \\ 
\cline{2-6}
&White Button & $100$ \% & $0.69 \pm 0.008$ s & $100$ \% & $1.32 \pm 0.023$ Nm \\ 
\midrule
Grab & HDMI Switch & $100$ \%  & $2.87 \pm 0.005$ s & N/A & N/A\\ 
\midrule
Place& HDMI Switch & $100$ \%  & $2.42 \pm 0.064$ s & N/A & N/A\\ 
\midrule
\multirow{2}*{Slide Object} & Wood & $100$ \% & $1.21 \pm 0.008$ s & $100$ \% & $8.34 \pm 0.125$ Nm \\ 
\cline{2-6}
&Cloth & $100$ \% & $1.21 \pm 0.005$ s & $100$ \% & $8.69 \pm 0.098$ Nm \\ 
\cline{2-6}
&Foil & $100$ \% & $1.21 \pm 0.005$ s & $100$ \%  & $9.56 \pm 0.077$ Nm \\ 
\midrule
\multirow{2}*{Drag} & Wood& $100$ \% & $1.02 \pm 0.06$ s & $100$ \% & $6.42 \pm 0.04$ Nm \\ 
\cline{2-6}
&Cloth & $100$ \% & $1.09 \pm 0.06$ s & $100$ \% & $6.5 \pm 0.016$ Nm \\ 
\cline{2-6}
&Foil & $100$ \% & $1.01 \pm 0.008$ s & $100$ \% & $11.86 \pm 0.078$ Nm \\ 
\midrule
%\multirow{2}*{Shove} & Wood& $100$ \% & $1.28 \pm 0.007$ s & $100$ \% & $4.2 \pm 0.19$ Nm \\ 
%\cline{2-6}
%&Cloth & $100 \%$ & $1.29 \pm 0.013$ s & $100 \%$ & $4.47 \pm 0.195$ Nm \\ 
%\cline{2-6}
%&Foil & $100 \%$ & $1.29 \pm 0.02$ s & $100 \%$ & $7.31 \pm 0.358$ Nm \\ 
%\midrule
\multirow{2}*{Cut} & Carton & $100$ \% & $1.69 \pm 0.009$ s & $100$ \% & $7.16 \pm 0.502$ Nm \\
\cline{2-6}
&Cloth & $100$ \% & $1.75 \pm 0.089$ s & $100$ \% & $6.08 \pm 0.105$ Nm \\ 
\cline{2-6}
&Foil & $100$ \% & $1.72 \pm 0.011$ s & $100$ \% & $5.92 \pm 0.145$ Nm \\ 
\midrule
Turn Mechanism & Key & $95$ \% & $0.71 \pm 0.19$ s & $100$ \% & $0.94 \pm 0.345$ Nm \\ 
\midrule
Swipe & Tablet & $100$ \% & $1.4 \pm 0.17$ s & $66$ \% & $3.1 \pm 0.155$ Nm \\ 
\midrule
Move Mechanism & Red Lever & $100$ \% & $1.33 \pm 0.036$ s & $86$ \% & $4 \pm 0.079$ Nm \\
\midrule
Bend & Cables & $100$ \% & $1.99 \pm 0.006$ s & $100$ \% & $4.37 \pm 0.195$ Nm \\ 
\midrule
Slide off & Battery Case & $98$ \% & $1.13 \pm 0.66$ s & $96$ \% & $7.22 \pm 0.623$ Nm \\ 
\midrule
\end{tabular}
}
\end{table*}
 
The implementation of $28$ tactile skills exhibits a robust behaviour, and high performance in various industrial automation tasks that have been subjected to significant process disturbances.
Importantly, the simple transfer of policies and the efficient learning through the parameter vector $\params$ demonstrate the versatility enabled by the taxonomy.

\begin{figure}
    \centering
    \includegraphics[width=\textwidth]{figures/experiments/energy_comparison.png}
    \caption{A comparison of the required energy to learn a great number of skills. The deep deterministic policy gradient (DDPG) algorithm is compared with the \skillmodelabbr{} framework (see Sec.~\ref{ch:architecture:mios:ggtwrep} both with and without transfer learning.}
    \label{fig:experiments:taxonomy:comparison}
\end{figure}

Surprisingly, the seeming disadvantage of having to design a large number of different skills is vastly mitigated by the fact that the vast majority of policies can be transferred without modifications within the same skill class.
The same policy was used for each skill class, respectively, and only the parameters $\params=\left[\params_\pi,\params_c\right]$ needed to be adapted (or relearned) to find the new optimum.
Some policies are even directly transferable between different classes.
When looking at the building blocks of the selected policies, one might even say that many manipulation processes can be solved by using a small toolset of building blocks.
This result supports the idea that the proposed approach is versatile enough to be relevant for realistic scenarios.

As this approach enables the learning of a wide range of skills in realistic settings, the issue of energy consumption in real-world 24/7 skill acquisition settings becomes a pertinent concern.
Therefore, the computational energy required for the presented approach is compared with that of an exemplary state-of-the-art deep learning system, as is illustrated in Figure \ref{fig:experiments:taxonomy:comparison}.
The \skillmodelabbr{} framework used in this work is compared with the deep deterministic policy gradient method (DDPG) \cite{lillicrap2015continuous}.
Furthermore, \skillmodelabbr{} is shown for the case in which every skill is learned from scratch as well as for the case in which learned solutions can be reused and transferred to subsequent learning.
This suggests that using current state-of-the-art data-based methods to learn many skills may require significant resource demands, as was anticipated, for example, in \cite{thompson2020computational}. However, using the \skillmodelabbr{} framework requires an order of magnitude lower energy, and even significantly less energy than that with transfer learning.

For the deep deterministic policy gradient (DDPG) framework, convergence was observed after $\sim300$ trials.
However, contrary to previous expectations, no robust solution was achieved most of the time.
A trial includes executing a skill and calculating the learning algorithm.
Learning a skill with the use of prior knowledge from another skill within the same subclass (level-$1$ transfer) reduces the number of required trials to $\sim30$ for \skillmodelabbr{}, and to $\sim60$ for transfer across subclasses (level-$2$ transfer).
DDPG, in its current form, does not achieve a reliable transfer.
To compare all calculations with the same widely available computing platform, an Intel NUC with an i7 processor was used that has an average power consumption of $30$ W.
To compare the power consumption for the different frameworks, only the CPU time allocated for the learning algorithm was considered, which is $1$ ms for \skillmodelabbr{} and $0.6$ s for DDPG.
For a conservative analysis, it is assumed that the CPU worked at full capacity in both approaches.
Consequently, the energy consumption per trial is estimated to be $0.03$~J for \skillmodelabbr{} and $18$~J for DDPG.
The overall energy consumption is
\begin{equation}
    e_{n_s}=\sum_{i=0}^{n_s} \boldsymbol{d}_i,
\end{equation}
where $e_{n_s}$ denotes the energy consumption for $n_s$ number of sequentially learned skills, and $\boldsymbol{d}_i$ is the vector for the required number of trials for each learned skill.
To make a reasonable comparison with transfer learning, it is assume that the goal is to sequentially learn $100$ skill subclasses with $10$ instances for each of a single skill class.



% ===================================================================================================
%                                                 |                                                 |
% -------------------------------------------- SECTION ---------------------------------------------|
%                                                 |                                                 |
% ===================================================================================================
\section{Thesis Structure}
%The thesis is structured as follows.
Chapter \ref{ch:foundations} provides the theoretical foundations for tactile skills, their synthesis from formal process definitions, the taxonomy of manipulation skills, skill learning, and planning.
Chapter \ref{ch:architecture} introduces the \skillmodelabbr{} framework and its implementation, the \software{} (\softwareabbr{}).
Furthermore, it presents numerous validation experiments such as related publications and demonstrators shown at public events that verify the overall learning architecture.
In Ch. \ref{ch:experiments} the theoretical foundations are experimentally validated using the architecture.
It presents experiments and results from skill synthesis, skill learning, transfer learning, and assembly planning.
Finally, Ch. \ref{ch:conclusion} concludes the thesis and gives an outlook on future work.


% ===================================================================================================
%                                                 |                                                 |
% -------------------------------------------- SECTION ---------------------------------------------|
%                                                 |                                                 |
% ===================================================================================================
\section{Curriculum Vitae}
%\begin{cv}{}
\begin{cvlist}{Education}
\item[Since 04/2018] \textbf{Dissertation}\\
Technical University Munich
\item[09/2014--03/2018] \textbf{Dissertation}\\
Gottfried Willhelm Leibniz Universit\"at Hannover
\item[10/2012--07/2014] \textbf{Master of Science}\\
Gottfried Willhelm Leibniz Universit\"at Hannover
\item[10/2007--09/2012] \textbf{Bachelor of Science}\\
Gottfried Willhelm Leibniz Universit\"at Hannover
\item[08/2003--07/2007] \textbf{High School Diploma}\\
Fachgymnasium Wirtschaft (Wunstorf)
\end{cvlist}

\begin{cvlist}{Professional Experience}
\item[Since 04/2022] \textbf{Franka Emika GmbH}\\
Head of Robotics Learning
\item[04/2018--03/2022] \textbf{Research Assistant}\\
Technical University Munich
\item[09/2014-03/2018] \textbf{Research Assistant}\\
Gottfried Willhelm Leibniz Universit\"at Hannover
\end{cvlist}

\begin{cvlist}{KPI Statistics}
    \item[Citations] $512$
    \item[h-index] $9$
    \item[i10-index] $7$
\end{cvlist}

\end{cv}

\subsection*{Publications}

\subsubsection*{Conference Papers}
\begin{itemize}
    \item \textbf{Johannsmeier, L.}, \& Haddadin, S. (2022, October). Can we reach human expert programming performance? A tactile manipulation case study in learning time and task performance. In Proc. International Conference on Intelligent Robots and Systems (IROS) (pp. 12081-12088). IEEE.
    \item Chen, X., \textbf{Johannsmeier, L.}, Sadeghian, H., Shahriari, E., Danneberg, M., Nicklas, A., ... \& Haddadin, S. (2022, October). On the Communication Channel in Bilateral Teleoperation: An Experimental Study for Ethernet, WiFi, LTE and 5G. In Proc. International Conference on Intelligent Robots and Systems (IROS) (pp. 7712-7719). IEEE.
    \item Voigt, F., \textbf{Johannsmeier, L.}, \& Haddadin, S. (2020). Multi-Level Structure vs. End-to-End-Learning in High-Performance Tactile Robotic Manipulation. In Proc. Conference on Robot Learning (CoRL) (pp. 2306-2316).
    \item Grischke, J., \textbf{Johannsmeier, L.}, Eich, L., \& Haddadin, S. (2019, May). Dentronics: review, first concepts and pilot study of a new application domain for collaborative robots in dental assistance. In Proc. International Conference on Robotics and Automation (ICRA) (pp. 6525-6532). IEEE.
    \item Kuhn, J., Ringwald, J., Schappler, M., Johannsmeier, L., \& Haddadin, S. (2019, May). Towards semi-autonomous and soft-robotics enabled upper-limb exoprosthetics: first concepts and robot-based emulation prototype. In Proc. International Conference on Robotics and Automation (ICRA) (pp. 9180-9186). IEEE.
    \item \textbf{Johannsmeier, L.}, Gerchow, M., \& Haddadin, S. (2019, May). A framework for robot manipulation: Skill formalism, meta learning and adaptive control. In Proc. International Conference on Robotics and Automation (ICRA) (pp. 5844-5850). IEEE.
    \item Haddadin, S., \& Johannsmeier, L. (2018, October). The art of manipulation: Learning to manipulate blindly. In Proc. International Conference on Intelligent Robots and Systems (IROS) (pp. 1-9). IEEE.
    \item Grassmann, R., \textbf{Johannsmeier, L.}, \& Haddadin, S. (2018, October). Smooth Point-to-Point Trajectory Planning in $ SE $(3) with Self-Collision and Joint Constraints Avoidance. In Proc. International Conference on Intelligent Robots and Systems (IROS) (pp. 1-9). IEEE.
    \item Shahriari, E., \textbf{Johannsmeier, L.}, \& Haddadin, S. (2018, June). Valve-based virtual energy tanks: A framework to simultaneously passify controls and embed control objectives. In Proc. American Control Conference (ACC) (pp. 3634-3641). IEEE.
    \item Haddadin, S., \textbf{Johannsmeier, L.}, Becker, M., Schappler, M., Lilge, T., Haddadin, S., ... \& Parusel, S. (2018, March). roboterfabrik: a pilot to link and unify German robotics education to match industrial and societal demands. In Proc. International Conference on Human-Robot Interaction (pp. 375-375). IEEE.
\end{itemize}

\subsubsection*{Journals}
\begin{itemize}
    \item Ringwald, J., Schneider, S., Chen, L., Knobbe, D., \textbf{Johannsmeier, L.}, Swikir, A., \& Haddadin, S. (2023). Towards Task-Specific Modular Gripper Fingers: Automatic Production of Fingertip Mechanics. Robotics and Automation Letters (R-AL), 8(3), 1866-1873.
    \item Haddadin, S., Parusel, S., \textbf{Johannsmeier, L.}, Golz, S., Gabl, S., Walch, F., ... \& Haddadin, S. (2022). The franka emika robot: A reference platform for robotics research and education. IEEE Robotics \& Automation Magazine (RAM), 29(2), 46-64.
    \item Naceri, A., Elsner, J., Tr\"obinger, M., Sadeghian, H., \textbf{Johannsmeier, L.}, Voigt, F., ... \& Haddadin, S. (2022). Tactile Robotic Telemedicine for Safe Remote Diagnostics in Times of Corona: System Design, Feasibility and Usability Study. Robotics and Automation Letters (R-AL), 7(4), 10296-10303.
    \item Grischke, J., \textbf{Johannsmeier, L.}, Eich, L., Griga, L., \& Haddadin, S. (2020). Dentronics: Towards robotics and artificial intelligence in dentistry. Dental Materials, 36(6), 765-778.
    \item Shahriari, E., \textbf{Johannsmeier, L.}, Jensen, E., \& Haddadin, S. (2019). Power flow regulation, adaptation, and learning for intrinsically robust virtual energy tanks. Robotics and Automation Letters (R-AL), 5(1), 211-218.
    \item Haddadin, S., \textbf{Johannsmeier, L.}, \& Ledezma, F. D. (2018). Tactile robots as a central embodiment of the tactile internet. Proceedings of the IEEE, 107(2), 471-487.
    \item \textbf{Johannsmeier, L.}, \& Haddadin, S. (2016). A hierarchical human-robot interaction-planning framework for task allocation in collaborative industrial assembly processes. Robotics and Automation Letters (R-AL), 2(1), 41-48.
\end{itemize}

\subsubsection*{Workshops}
\begin{itemize}
    \item \textbf{Johannsmeier, L.}, Ringwald, J., Kuehn, J., \& Haddadin, S. (2016). Teleoperated semi-autonomous control of the lwr and a humanoid hand via the myo armband. Workshop in International Conference on Robotics and Automation (ICRA), 12, 13.
    \item \textbf{Johannsmeier, L.}, \& Haddadin, S. (2015). A framework for task allocation in collaborative industrial assembly processes. In 8th International Workshop on Human-Friendly Robotics.
\end{itemize}

\subsubsection*{Book Chapters}
\begin{itemize}
    \item A\"smann, U., Chen, L., Ebert, S., G\"ohringer, D., Grzelak, D., Hidalgo, D., ..., \textbf{Johannsmeier, L.}, ... \& Podlubne, A. (2021). Human–robot cohabitation in industry. In Tactile Internet (pp. 41-73). Academic Press.
\end{itemize}

\subsection*{Submission in Preparation}

\begin{itemize}
    \item \textbf{Johannsmeier, L.}, Schneider, S., Voigt, F., \&Haddadin, S. (2023). Motor Memory for Few-Shot Learning in Robotic Manipulation. Submitted to Science Robotics.
    \item \textbf{Johannsmeier, L.}, Schneider, S., Li, Y., Burdet, E., \&Haddadin, S. (2023). Robot Curricula: Process-Centric Industrial Manipulation Taxonomy for Organisation, Classification and Synthesis of Tactile Robot Skills. Submitted to Nature Machine Intelligence.
\end{itemize}