In general, a taxonomy is a systematic approach to classify things in groups or types in a structured way and can significantly aid in providing further insights into the respective field often allowing for a wider perspective.
Many taxonomies have a hierarchical tree structure, however, this is not a requirement.
The most commonly known and oldest taxonomies are the classifications of organisms \cite{CavalierSmith.1998} which are constantly updated and revised as new information becomes available.
These taxonomies classify e.g. animals according to their relation to each other, nowadays also supported by DNA sequencing.
Other prominent examples are the classification of diseases \cite{Snider.2003}, which orders diseases according to their cause, pathogenesis or symptoms, of viruses \cite{Adams.2013} where viruses are classified by e.g. their morphology and nucleic acid type, and folk taxonomies \cite{Park.2003} which order e.g. animals and plants according to social knowledge and everyday language embedded in specific cultures.
In the economic domain corporate taxonomies hierarchically classify physical or conceptual entities such as products, processes, documents, or job titles \cite{DelphiGroup.2004}.
The human factors and classification system \cite{Wiegmann.2017} is a safety taxonomy designed to identify the human cause of accidents, which was primarily motivated by the rate of human error related to flight accidents.


The taxonomy introduced in this thesis is concerned with manipulation skills for autonomously acting robots.
In the following related work in this area is reviewed and followed with a review on taxonomies in robotics in general.

\subsubsection{Skill Taxonomies}

Extensive real-world-tested manipulation skill taxonomies are difficult to build since they require a significant effort in terms of programming, experimentation and validation.
In \cite{Bloomfield.2003} the authors describe a taxonomy of haptic tasks based on human manipulation which also relates to robotics. First, they categorize manipulation action by broad labels such as significant required arm strength or the need of two-handed manipulation. Then, they added a second category that is determined by the direction of force and/or torque used in the respective task.
In \cite{Huckaby.2012} an assembly taxonomy is proposed which represents the decomposition of complex assembly tasks into simple skills which can be reused in order to reduce programming time and overhead.
Another approach is shown in \cite{Leidner.2015}. The authors combine high-level AI-planning methods and compliant manipulation schemes in order to classify typical household shores. They created a hierarchical taxonomic structure that classifies manipulation tasks with respect to contact complexity. Their taxonomy makes use of general contact classifiers to make it accessible to high-level planning and introduces a sub-categorization based on the detailed contact situation to further distinguish manipulation actions.
There has also been work that aims more at the integration of robots into established frameworks. In \cite{Bjorkelund.2011} the authors discuss the compromise between standards and flexible formalisms in industry. Although they do not make use of a taxonomic approach they argue for the use of flexible skills and their connection to context-dependent data. In \cite{Pfrommer.2013} skills are derived from an abstract process view and defined within the concept of PPR in AutomationML. Skills are understood as more general abilities and tasks as concrete instantiations.
Although not directly a taxonomy of manipulation skills, \cite{HuamanQuispe.2018} introduces a system of benchmark tasks categorized by the type of test and the tested part of the robot.
They furthermore provide guidelines for designing test protocols and experiments.
In \cite{Paulius.2019} a taxonomy of motions related to cooking tasks is developed.
The motions are grouped into similar sets based on trajectory and contact attributes.


\subsubsection{Other Taxonomies in Robotics}

Among the most prominent works on taxonomic structures in robotics and related fields are grasp taxonomies.
Those taxonomies do not directly relate to the work in this paper but provide an important source of inspiration in terms of structure.
They usually classify different grasps in a hierarchical structure for human or anthropomorphic hands.
Most notably among them are the classifications of Cutkosky \cite{Cutkosky.1989} and Bullock \cite{Bullock.2013}.
In \cite{Liu.2014} these taxonomies are extended by classification elements related to tactile interaction.
One of the first approaches towards such a taxonomy is \cite{Napier.1956} that divided grasps into power and precision grasps.
In \cite{Dudek.1993} a taxonomy is presented which contains various approaches to structuring swarm robots.
In the field of multi-robot system, the authors of \cite{Gerkey.2004} proposed a taxonomy of task allocation where it is shown that many problem related to multi-robot task allocation can be viewed as instances of other, well studied, optimization problems.
It is demonstrated how the taxonomy can be used to synthesize new approaches based on existing theory.
This taxonomy is extended by further research in \cite{Korsah.2013}.
Taxonomic research has also been done e.g. in humanoid robotics. In \cite{Borras.2015,Borras.2017} a taxonomy for whole-body poses and the transitions between them has been developed. The authors propose a formal definition to characterize whole-body poses and to provide a framework for planning and benchmarking whole-body motion and loco-manipulation.
In \cite{Tsiakas.2018} a taxonomy of robot-assisted training systems is introduced that classifies existing systems by e.g. requirements, interaction types, level of autonomy and possibilities of personalization.
One result of this taxonomy is a set of open challenges for designing and developing such training systems.
A more general taxonomy of robotic hardware systems is proposed in \cite{Linjawi.2018}.
The authors developed a representation of robots that links application requirements to robot capabilities with the aim of aiding application developers in choosing the right platform for their purpose.
\cite{Fong.2003} introduced a more specific taxonomy of socially interactive robots.

A taxonomy of action representations in robotics is given in \cite{Zech.2019}.
In \cite{BautistaBallester.2014} a taxonomic overview of programming by demonstration approaches is developed.
The authors reviewed existing work and identified the basic elements of current methods.
In the field of human-robot interaction \cite{Onnasch.2021} introduced a taxonomy that classifies human-robot interaction scenarios by e.g. the role of the human, communication channels, proximity or field of application.