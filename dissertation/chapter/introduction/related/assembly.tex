Human-robot collaboration \cite{Bauer.2008,Goodrich.2008,Chandrasekaran.2015,Ajoudani.2018} has developed into a large field involving various research directions such as human-robot interaction \cite{Argall.2010,Villani.2018}, safety \cite{Maurtua.2017,Arents.2021,Valori.2021}, and (multi-agent) task planning and allocation \cite{Rosell.2004,Jia.2013,Alatartsev.2015,Khamis.2015,tsarouchi2016human}.
Assembly planning is highly related to human-robot collaboration since human workers and robots are often utilized as a team.
Numerous works have built entire frameworks to solve assembly problems.
\cite{wang2019symbiotic} provides an overview on human-robot collaborative assembly and the used methods.
To represent assembly planning problems on task level in robotics, various approaches have been used such as AND/OR graphs \cite{HomemdeMello.1990} and Petri nets \cite{zhang1989representation}.
Other works focus the planning on motion level such as \cite{bonert2000motion} that presents an approach that solves an assembly problem on motion level through the use of a genetic algorithm.
The ROBO-PARTNER project's aim was to develop intelligent assembly cells, task planning, and communication methods for collaborative human-robot assembly \cite{michalos2014robo}.
\cite{Darvish.2018} shows a multi-layered framework called flexHRC which also uses AND/OR graphs to model human-robot cooperation models with an additional focus on the use of wearable sensors for additional data on the environment and human co-workers.
\cite{ying2021cyber} introduces a cyber-physical assembly system-based metaheuristic for automated assembly sequence planning.
\cite{rodriguez2020pattern} developed a method to transfer knowledge about constraints from one assembly onto another.