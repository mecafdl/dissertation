%\begin{enumerate}
%\item What has been written on your topic
%\item Who the key authors are and what the key works are
%\item The main theories and hypotheses
%\item \textbf{The main themes that exist in the literature.}
%\item \textbf{Gaps and weaknesses that your study will then help fill.} This thesis emphasizes the importance of online learning and thus presents methods that cast and extends typical problems in robot system identification as online learning problems where the parameters of a robot are learned as data arrives. The properties of the body (body schema) are learned from motion data and scarce prior information in the form of first-order principles for kinematics and dynamics.
%\end{enumerate}


%A directly related approach to the work in this dissertation is presented in \cite{Lutter2023InductiveBiasesMachine} where the question \say{How can one combine existing knowledge and data-driven deep learning methods to learn models and policies applicable to physical robots?} is addressed.\\

One fundamental characteristic of humans is the awareness of the body, which allows its control with dexterity and plasticity. We know where our body is in space and we know how to take it from its present situation at time $t$, its current \textit{state} $\bm{s}_t$, to a predicted or expected future state $\bm{s}_{t+1}$ at time $t+1$ given a carefully chosen \textit{action} $\bm{a}_t$. The ability to anticipate the future and to determine the actions that can take us there is connected to the notions of forward and inverse models in cognitive science and robotics \cite{Kawato1999Internalmodelsmotor,Pierella2019dynamicsmotorlearning,NguyenTuong2011Modellearningrobot}. An alternative and complementary interpretation is that of sensorimotor contingencies \cite{Maye2013Extendingsensorimotorcontingency,Jacquey2019Sensorimotorcontingenciesas,Buhrmann2013dynamicalsystemsaccount}, which denote the structured relations between the actions of an agent and the ensuing sensory inputs resulting from interaction. The potential sensorimotor interactions are connected to the concept of \textit{embodiment}, which, as expressed by Pfeifer et al. \cite{Pfeifer2006Howbodyshapes}, deals with \say{how the body shapes the way we think,} putting a premium on the morphology and capacities of an agent. Last, a few strongly related concepts are pertinent to the discussion: the \textit{ecological self} \redtext{REFERENCE}, the \textit{physical self} \redtext{REFERENCE}, and the \textit{body schema} \redtext{REFERENCE}. They are different interpretations of \textit{body models} that describe its physical structure and capabilities, and integrate sensory information for prediction and action. Research in cognitive science tells us that we construct and maintain representations involving the previous concepts from the moment that we are in womb and during all our conscious existence. Maintaining those models is not only a matter of refining and tuning them but also may involve adapting them to accommodate unexpected drastic changes. The level of bodily awareness encoded in those representations allows to achieve remarkable feats, like reach for things without even looking, catching an object in midair, or contorting our bodies to maneuver or bodies in tight spaces.

As discussed in the \nameref{ch:introduction} the vision for future robots involves giving them bodily awareness capabilities inspired in humans. To contextualize what is the state of the art towards realizing that vision, this chapter provides a review and discussion of relevant works that address the field of modeling in robotics; in particular, model learning. The discussion will include a brief review of the fundamental methods to model robots and will gradually move to the paradigm of learning local and global models with and without prior knowledge. The discussion will connect with the above mentioned concepts pertaining body models and present pertinent works that have presented methods to learn the body schema and sensorimotor representations in robotics

\section{Model learning in robotics}
\subsection{Classical and recent works in system identification}
\subsection{Local and global models linear models}
\subsection{End-to-end learning (black box models)}
\section{Data-driven learning with structure information}
\section{Model learning and the body schema}
\subsection{Internal representations}
\subsection{Sensorimotor maps}
