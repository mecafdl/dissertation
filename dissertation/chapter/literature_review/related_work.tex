%\begin{enumerate}
%\item What has been written on your topic
%\item Who the key authors are and what the key works are
%\item The main theories and hypotheses
%\item \textbf{The main themes that exist in the literature.}
%\item \textbf{Gaps and weaknesses that your study will then help fill.} This thesis emphasizes the importance of online learning and thus presents methods that cast and extends typical problems in robot system identification as online learning problems where the parameters of a robot are learned as data arrives. The properties of the body (body schema) are learned from motion data and scarce prior information in the form of first-order principles for kinematics and dynamics.
%\end{enumerate}


%A directly related approach to the work in this dissertation is presented in \cite{Lutter2023InductiveBiasesMachine} where the question \say{How can one combine existing knowledge and data-driven deep learning methods to learn models and policies applicable to physical robots?} is addressed.\\

One fundamental characteristic of humans is the awareness of the body, which allows its control with dexterity and plasticity. We know where our body is in space, we know much of that space our body takes, and we know how to take it from its present situation at time $t$, its current \textit{state} $\bm{s}_t$, to a predicted or expected future state $\bm{s}_{t+1}$ at time $t+1$ given a carefully chosen \textit{action} $\bm{a}_t$. The ability to anticipate the future and to determine the actions that can take us there is connected to the notions of forward and inverse models in cognitive science and robotics \cite{Kawato1999Internalmodelsmotor,Pierella2019dynamicsmotorlearning,NguyenTuong2011Modellearningrobot}. An alternative and complementary interpretation is that of sensorimotor contingencies \cite{Maye2013Extendingsensorimotorcontingency,Jacquey2019Sensorimotorcontingenciesas,Buhrmann2013dynamicalsystemsaccount}, which denote the structured relations between the actions of an agent and the ensuing sensory inputs resulting from interaction. The potential sensorimotor interactions are connected to the concept of \textit{embodiment}, which, as expressed by Pfeifer et al. \cite{Pfeifer2006Howbodyshapes}, deals with \say{how the body shapes the way we think,} putting a premium on the morphology and capacities of an agent. Last, a few strongly related concepts are pertinent to the discussion: the \textit{ecological self} \redtext{REFERENCE}, the \textit{physical self} \redtext{REFERENCE}, and the \textit{body schema} \redtext{REFERENCE}. They are different interpretations of \textit{body models} that describe its physical structure and capabilities, and integrate sensory information for prediction and action. Research in cognitive science tells us that we construct and maintain representations involving the previous concepts from the moment that we are in womb and during all our conscious existence. Maintaining those models is not only a matter of refining and tuning them but also may involve adapting them to accommodate unexpected drastic changes. The level of bodily awareness encoded in those representations allows to achieve remarkable feats, like reach for things without even looking, catching an object in midair, or contorting our bodies to maneuver in tight spaces or negotiate obstacles.

As discussed in the \nameref{ch:introduction} the vision for future robots involves giving them bodily awareness capabilities inspired in humans. To contextualize what is the state of the art towards realizing that vision, this chapter provides a review and discussion of relevant works that address the field of modeling in robotics; in particular, model learning. The discussion will include a brief review of the fundamental methods to model robots and will gradually move to the paradigm of learning local and global models with and without prior knowledge. The discussion will connect with the above mentioned concepts pertaining body models and present pertinent works that have presented methods to learn the body schema and sensorimotor representations in robotics

\redtext{It is also important to emphasize internal models for awareness \cite{Holland2003RobotsInternalModels}}

% ===========================================================================================
%                                           |                                               |
% -------------------------------------- SECTION -------------------------------------------|
%                                           |                                               |
% ===========================================================================================
\section{Model learning in robotics}
Existing learning frameworks often focus on developing forward and inverse models using either a global or local approach to capture input-output relationships \cite{NguyenTuong2011Modellearningrobot}.

Global methods risk overfitting and computational overload, while local methods suffer from limited generalization and hyperparameter sensitivity \cite{Thrun2002Probabilisticrobotics,Goodfellow2016DeepLearning}. Despite advancements in computational power and data availability, deep learning faces challenges due to the neglect of prior principled knowledge, making it difficult to determine dedicated neural network architectures \cite{Baker2017Designingneuralnetwork,Elsken2019Neuralarchitecturesearch}. 

\subsection{End-to-end learning}

\subsubsection{Classical neural networks model learning}
Model-learning problems using neural networks (NN) mainly involves:
\begin{enumerate}
	\item Input/output data collection: assumed to be available
	\item Architecture design: usually found by trial and error
	\item Parameter optimization/learning: via well understood schemes, e.g., backpropagation and variants
\end{enumerate}
Designing NN for a particular problem requires experts to determine the best topology, i.e. the number of nodes and layers, connectivity, and activation functions \cite{Matteucci2006ELeaRNTEvolutionarylearning}. Furthermore, generalization is difficult as the architecture needs to balance achieving accuracy while avoiding overfitting \cite{Rocha2005Simultaneousevolutionneural,He2015Topologicaloptimisationartificial,Matteucci2006ELeaRNTEvolutionarylearning,Kwok1995Constructivefeedforwardneural,Lawrence1998Whatsizeneural,Talebi2010NeuralNetworkBased}. Therefore, if a NN is used without any model information, large amounts of training data are required to generalize to unknown data \cite{Urolagin2012Generalizationcapabilityartificial}.

\subsubsection{Topology learning related works}
Finding NN topologies is an important and challenging step \cite{Tirumala2016Evolvingdeepneural,Rocha2005Simultaneousevolutionneural,Baker2017Designingneuralnetwork}. Normally, function approximation via NN uses empirical topologies that rely on numerous parameters and do not lend themselves to interpretation. Such models provide no insight into the actual relation between the system variables. Recent works have aimed to find optimal topologies automatically. For example, evolutionary methods have been utilized to optimize the topology of Feed Forward NN (FFNN) \cite{Rocha2005Simultaneousevolutionneural,Matteucci2006ELeaRNTEvolutionarylearning} as well as  deep NN \cite{Tirumala2016Evolvingdeepneural}, by adding/deleting connections and weights. Constructive methods \cite{Kwok1995Constructivefeedforwardneural} and pruning methods \cite{Srinivas2016LearningNeuralNetwork} have also been applied to FFNN. Another method used for FFNN represents the network as a graph and reduces its degrees-of-freedom (DoF) \cite{He2015Topologicaloptimisationartificial}. Furthermore, reinforcement learning (through Q-learning) and topology learning (using variance analysis) have also been implemented to generate architectures \cite{Baker2017Designingneuralnetwork,Castillo2007Functionalnetworktopology}. Noticeably, for learning complex dynamical systems, such as articulated robot structures, results have been limited in accuracy and generalization capabilities \cite{NguyenTuong2011Modellearningrobot,NguyenTuong2008Learninginversedynamics,NguyenTuong2010Usingmodelknowledge}. 

\subsubsection{Robot inverse dynamics estimation via classical NN}\label{sec:classic_inv_dyn}
NN have been applied in numerous variants to model robot inverse dynamics. In \cite{Atencia2015Hopfieldnetworksoptimization}, Hopfield NN were applied to identify the inertial parameters. Likewise, in \cite{Zhu2014Inertiaparameteridentification} a FFNN that used the regressor matrix as training samples was applied. Extreme Learning Machines were utilized in \cite{Bargsten2016ExperimentalRobotInverse} with the same purpose. More recently a two-hidden-layers network with rectified linear activation units (ReLU) was used in \cite{Christiano2016TransferSimulationReal}. Similarly, recurrent NN have been used to account for the sequential nature of the data. In \cite{Yan1997Robotlearningcontrol}, a recurrent NN in the hidden layer of an otherwise conventional three-layer FFNN was proposed. Additionally, self-organizing-networks, in conjunction with echo state networks, were used in \cite{Polydoros2015Realtimedeep} via a real-time deep learning algorithm.
% ===========================================================================================
%                                           |                                               |
% -------------------------------------- SECTION -------------------------------------------|
%                                           |                                               |
% ===========================================================================================
\section{Data-driven learning with structure information}
In lieu of the challenges faced by deep learning to capture the intricacies of complex systems from scratch,  
Issues such as low sample efficiency, extended training times, and limited generalization highlight the necessity of balancing data-driven and principle-driven approaches \cite{Pierson2017Deeplearningrobotics,Suenderhauf2018limitspotentialsdeep}. Recently, there has been a growing acknowledgment of the importance of integrating structure into the learning of physical systems \cite{Geist2021Structuredlearningrigid,Lutter2023Combiningphysicsdeep}.

% ===========================================================================================
%                                           |                                               |
% -------------------------------------- SECTION -------------------------------------------|
%                                           |                                               |
% ===========================================================================================
\section{Model learning and the body schema}




Building on the significance of structure in model learning for robotics, this dissertation addresses the inference of essential morphological properties in tree-like floating base structures, mimicking the development of a body schema. Efforts in cognitive robotics stress the pivotal role of internal body models in enhancing spatial awareness, motor control, and adaptability \cite{Nguyen2021Sensorimotorrepresentationlearning,Hoffmann2010Bodyschemarobotics}. However, consensus is lacking on what constitutes a robot's body schema. Some approaches focus solely on learning the kinematic structure, relying predominantly on off-body vision \cite{Hersch2008Onlinelearningbody,MartinezCantin2010Bodyschemaacquisition,Hart2011roboticmodelecological,Lipson2019Taskagnosticself,Chen2022Fullybodyvisual,Sturm2009Bodyschemalearning}. Others explore sensorimotor associations between proprioceptive, tactile, and visual modalities \cite{Fuke2007BodyImageConstructed,Malinovska2022connectionistmodelassociating,Nguyen2019Reachingdevelopmentvisuo,Pugach2019BrainInspiredCoding,Lanillos2016Yieldingselfperception}, but they provide limited insights into the robot's physical structure.

Model-based robotics offers reliable methods for identifying physical attributes of robots based on known mechanical topologies. Conventional calibration routines \cite{Hollerbach1996CalibrationIndexTaxonomy} and offline system identification methods \cite{Swevers2007Dynamicmodelidentification,LeboutetInertialParameterIdentification} are effective for known kinematic structures in controlled environments. However, these methods face challenges when applied to floating base robots without standardized identification procedures \cite{Ayusawa2014Identifiabilityidentificationinertial,Lee2022OptimizedSystemIdentification}. Importantly, these conventional methods were not initially designed for integration into online learning frameworks. While model-based robotics addresses kinematic calibration and forward/inverse kinematics, it provides limited insights into the comprehensive understanding of joint and link arrangement, known as mechanical topology. In cognitive robotics, only a few studies have approached this problem for self-modeling and monitoring, relying on exteroceptive vision \cite{Bongard2006Automatedsynthesisbody,Bongard2006Resilientmachinescontinuous}. Regardless of the approach taken---black-box machine learning, cognitive methods, or model-based robotics---reliance on external measurement devices persists, overlooking embodied sensing modalities.

In summary, current robotics research reveals gaps in understanding and methods for refining body models. A comprehensive interpretation of the robot body schema and the determination of essential features are crucial. Identifying the fundamental set of necessary signals, both proprioceptive and embodied exteroceptive, is paramount. Integrating advanced machine learning with prior information and first-order principles shows promise for enhanced body models, addressing data requirements and generalization issues. However, the lack of synergy between modeling and learning approaches, along with the absence of a unified scheme for relevant learning stages, represents notable gaps requiring attention to advance robotics into more sophisticated and adaptable embodied systems.

\section{Model learning in robotics}
\subsection{Classical and recent works in system identification}
\subsection{Local and global models linear models}
\subsection{End-to-end learning (black box models)}
\section{Data-driven learning with structure information}
\section{Model learning and the body schema}
\subsection{Internal representations}
\subsection{Sensorimotor maps}
