As a next step, the taxonomy's set of skills should be extended and experimentally validated.
The current taxonomy has a strong focus on processes from machine tending and assembly.
An extension could focus on material manipulation such as sawing, filing, and grinding. A thorough experimental investigation on those processes would bring new insights that might lead to improvements to the taxonomy hierarchy, the synthesis process, and the \skillmodelabbr{} model.
Moreover, the validation experiments can be improved by considering more realistic process disturbances to test against robustness and performance stability.
Besides adding more processes to the taxonomy, it would also be a promising next step to introduce new hardware platforms such as different grippers and arms.
This may not only lead to a generalization of the theoretical foundations and capabilities of the overall framework but also allows to include processes in the taxonomy that could so far not be handled due to hardware restrictions.
Beyond the extension to more content, the taxonomy may also be adapted to be used as a programming interface.
To achieve this, the process definitions have to be formulated as program blocks which can be parameterized by process experts without the need for robot knowledge.
Another interesting next step is the use of more complex policies within the \skillmodelabbr{} models such as dynamic motion primitives (DMP), or neural networks. This could combine very general policy representations with a framework that guarantees stable control and safety.
The upscaling of the (transfer) learning capabilities to many robots and skills may accelerate the overall learning performance beyond what is possible today.
There have been parallel investigations into collective learning capabilities on the basis of the results of this work.
Simulations and experiments indicate significant potential to increase the learning performance for large-scale systems when large numbers of robots share their acquired knowledge.
The next steps into this direction involve setting up a large real-world experiment, the necessary software components and devising an experiment procedure. Results from such a system that support the already established theory would have considerable impact on research and in the long-term on commercial robotics products.
Finally, the results of this thesis in terms of tactile skill synthesis and learning may be of significant relevance in ongoing research and development in geriatronics.
The envisioned humanoid assistants have to rely on a vast range of manipulation skills in order to successfully interact with their highly dynamic and potentially unknown environment.
This opens up the possibilities to extend the work of this thesis to other robot platforms (mobile and bimanual with anthropomorphic hands), and a new domain of processes.
