The work in this thesis has had an impact on technological, scientific, and industrial levels.
Numerous demonstrators have been developed and showcased at international conferences, trade fairs, and other public events.
Some of the most prominent examples are the automatica and Hannover Messe trade fairs, where the learning architecture and tactile skill capabilities have been shown to the highest political level of Germany which led to follow-up visits by e.g. the German chancellor and further investments into the research efforts of the Munich Institute of Robotics and Machine Intelligence.
Other significant public events were the official opening of the Munich Institute of Robotics and Machine Intelligence and the Falling Walls conference.
Additionally, parts of this work are used in the labs of Vodafone as part of a telepresence showcase.
Further demonstrations at various robotics conferences such as ICRA and IROS disseminated the results into the research community inspiring new work in the fields of manipulation learning and planning.
There has been a technology transfer of published algorithms into the industry as well as close collaboration.
Also, new projects have been initiated such as the KI.Fabrik, a lighthouse initiative in close cooperation with the industry.
Based on the theoretical foundations the \software (\softwareabbr{} was developed that enabled many other works at MIRMI and beyond.
It powers and enables experimental work on lab automation \cite{knobbe2022core}, automatic production \cite{ringwald2023towards}, learning, control, telepresence \cite{chen2022communication,moortgat2020feeling} and human-robot interaction.