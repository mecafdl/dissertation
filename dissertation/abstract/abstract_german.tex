%*****************************************************************************************
%*****************************************************************************************
%*****************************************************************************************
\chapter*{Zusammenfassung}

Roboter werden in unserer Gesellschaft immer mehr zu einem allt\"aglichen Werkzeug.
In der Industrie werden Roboter schon lange eingesetzt, aber auch im Gesundheitswesen, in der Logistik und im h\"auslichen Bereich werden solche flexiblen Automatisierungsl\"osungen zunehmend genutzt.
Die wichtigsten Treiber f\"ur diese Entwicklung sind der demografische Wandel und der Arbeitskr\"aftemangel.
Um den kommenden Anforderungen gerecht zu werden, sind Technologiespr\"unge notwendig, die den Weg f\"ur die n\"achste Generation taktiler Roboter ebnen.
Zuk\"unftige Arbeitspl\"atze und Fabriken werden sich auf die Vielseitigkeit und die Manipulationsf\"ahigkeiten von taktilen Robotern verlassen, um hochdynamische Fertigungsprozesse zu automatisieren und gleichzeitig den Energiebedarf niedrig zu halten.
Die Roboter werden im laufenden Betrieb L\"osungen f\"ur Probleme generieren, ohne dass eine manuelle Programmierung erforderlich ist, und diese v\"ollig autonom erlernen.
Sie werden zu intelligenten Werkzeugen in autonomen, modularen Fabriken, zu Assistenten in Haushalten und zu Verk\"orperungen f\"ur Telepr\"asenzanwendungen.
Neben Interaktion und Mobilit\"at sind die Manipulationsf\"ahigkeiten der Schl\"ussel zu weiteren Fortschritten in dieser Richtung.

Auf dem Gebiet der Robotermanipulation wurden, vor allem in den letzten zehn Jahren, bereits gro{\ss}e Fortschritte erzielt, und die Ergebnisse, wie z. B. neuartige Kontrollmethoden, Bewegungsplanung und Wahrnehmung, finden allm\"ahlich Eingang in den industriellen Bereich.
Heute gibt es bereits Anwendungsbereiche wie Testen, Inspektion und Maschinenbedienung, in denen fortgeschrittene, jedoch vorprogrammierte Roboterf\"ahigkeiten zum Einsatz kommen.
Betrachtet man jedoch schwierigere Manipulationsprozesse, so besteht eine zweiteilige Herausforderung. Erstens fehlt eine konstruktive Verbindung zwischen Prozessbeschreibungen und taktilen F\"ahigkeitsmodellen. Zweitens ist das Erlernen solcher Modelle f\"ur anspruchsvolle taktile Prozesse schwierig, und derzeit sind nicht einmal die leistungsst\"arksten (traditionellen) End-to-End-Frameworks in der Lage, zuverl\"assige L\"osungen f\"ur Manipulationsf\"ahigkeiten zu generieren, geschweige denn eine \"ubertragung zwischen verschiedenen Problemen zu erreichen.

Der Hauptbeitrag dieser Arbeit ist eine autonomes Synthese-Pipeline, welche formale Prozessdefinitionen, die von Prozessexperten bereitgestellt werden, mit kompatiblen taktilen F\"ahigkeitsmodellen verbindet.
Als Verbindung zwischen diesen Endpunkten wird eine Taxonomie der taktilen Manipulationsf\"ahigkeiten vorgeschlagen, die eine eindeutige L\"osungsf\"ahigkeit f\"ur einen bestimmten Prozess bestimmt.
Die taktile Fertigkeit wird dann als ein Lernproblem formuliert, das mit einer effizienten Lernarchitektur gel\"ost werden kann.
Dieser teile-und-hersche-Ansatz l\"asst sich in Bezug auf die Anzahl der Fertigkeiten und Prozesse unbegrenzt skalieren und kann auch mit automatischen Prozessplanungssystemen verbunden werden. Auf diese Weise k\"onnen komplexe Aufgaben gel\"ost werden, ohne dass eine Programmierung durch Experten erforderlich ist.
Die sich daraus ergebende Vielseitigkeit bei der L\"osung von Manipulationsprozessen macht dieses Konzept zu einem ersten Schritt in Richtung eines immer umfangreicheren Curriculums f\"ur Roboter, das, \"ahnlich wie etablierte Curricula f\"ur (menschliche) Auszubildende in industriellen Bereichen, einen Rahmen f\"ur den Erwerb aller relevanten Manipulationsf\"ahigkeiten f\"ur Roboter bieten k\"onnte.

Diese Arbeit liefert die theoretischen Grundlagen f\"ur taktile F\"ahigkeiten, ihre Synthese, Lern- und Planungsf\"ahigkeiten.
Sie beschreibt die Umsetzung der Grundlagen und eine Reihe von Validierungsf\"allen.
Es werden umfangreiche experimentelle Arbeiten vorgestellt, die die F\"ahigkeiten des Manipulationsrahmens demonstrieren.
Ausf\"uhrliche Verifizierungsexperimente validieren den Ansatz f\"ur eine aussagekr\"aftige Anzahl von Fertigkeiten und zeigen eine hohe Robustheit und Leistung, die grundlegende Anforderungen f\"ur industrielle Anwendungen sind.
Die Lernleistung der vorgeschlagenen Architektur und Synthesepipeline wird f\"ur modernste maschinelle Lernalgorithmen analysiert.
Das Gesamtsystem zeigt eine \"uberlegene Lernleistung und Effizienz im Vergleich zu den modernsten End-to-End-Ans\"atzen.
In einer umfangreichen experimentellen Kampagne wurde sogar ein systematischer Transfer-Lerneffekt zwischen verschiedenen anspruchsvollen physikalischen Manipulationsf\"ahigkeiten beobachtet. Dies erm\"oglicht das Erlernen einer gro{\ss}en Anzahl von Fertigkeiten durch systematische Ausnutzung ihrer \"Ahnlichkeit.
Die Leistung der optimierten Fertigkeiten und die Lernleistung werden direkt mit den menschlichen F\"ahigkeiten verglichen.
Die Ergebnisse zeigen, dass f\"ur einige Fertigkeiten bereits das Niveau menschlicher F\"ahigkeiten erreicht werden kann.
Abschlie{\ss}end wird anhand eines kollaborativen Montageplanungsproblems f\"ur industrielle Mechatroniksysteme mit engen Toleranzen und mehrdimensionalen Einf\"ugeprozessen die Planungsf\"ahigkeit des Frameworks demonstriert.
Mit dem entwickelten Framework ist es schlie{\ss}lich m\"oglich, die Roboterprogrammierung auch f\"ur komplexe Manipulationsprozesse zu automatisieren, ohne dass Roboterexpertenwissen erforderlich ist.

Die Ergebnisse dieser Arbeit haben weitere Forschungsarbeiten im Bereich der Steuerung, des Lernens, der Telepr\"asenz und der Bewegungsplanung beeinflusst und sogar ein v\"ollig neues Gebiet, n\"amlich die Dentronik, er\"offnet. Dar\"uber hinaus haben sie verschiedene Projekte, Ver\"offentlichungen und Produkte beeinflusst.
Nach bestem Wissen des Autors ist diese Arbeit die erste, die es Nicht-Experten erm\"oglicht, komplexe Manipulationsprobleme auf industrieller Ebene durch einen automatischen Synthese- und Lernansatz mit geringem Energiebedarf und nur begrenzten Rechenressourcen zu l\"osen.