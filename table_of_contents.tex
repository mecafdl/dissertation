\documentclass{article}
\usepackage{blindtext}
\usepackage{titlesec}
\title{Table of Contents}
\author{Fernando Diaz Ledezma}
\date{ }
\begin{document}
	
	\maketitle
	
	\tableofcontents
	
% ===============================================================================================
%                                             |                                                 |
%  --------------------------------------- SECTION ---------------------------------------------|
%                                             |                                                 |
% ===============================================================================================
\section{INTRODUCTION}
\subsection{Motivation}
\subsection{Problem Statement}
\subsection{Research Questions and Contribution}
%\subsection{Impact}
\subsection{Overview of content}

% ===============================================================================================
%                                             |                                                 |
%  --------------------------------------- SECTION ---------------------------------------------|
%                                             |                                                 |
% ===============================================================================================
\section{LITERATURE REVIEW}
\subsection{Model learning in robotics}
\subsubsection{Classical and recent works in system identification}
\subsubsection{Local and global models linear models}
\subsubsection{End-to-end learning (black box models)}
\subsection{Data-driven learning with structure information}
\subsection{Model learning and the body schema}
\subsubsection{Internal representations}
\subsubsection{Sensorimotor maps}

% ===============================================================================================
%                                             |                                                 |
%  --------------------------------------- SECTION ---------------------------------------------|
%                                             |                                                 |
% ===============================================================================================
\section{THEORETICAL FRAMEWORK}
\subsection{Robot equations of motion}
\subsubsection{Forward and inverse dynamics}
\subsubsection{The Newton-Euler formulation of the inverse dynamics}
%(a) The kinematics forward recursion
%(b) The dynamics backward recursion
\subsubsection{Composition of the kinematics and dynamics}
%\subsection{Basics of robot system identification}
%\subsubsection{Kinematic calibration}
%\subsubsection{Inertial parameter identification}
%(a) For fixed based robots
%(b) For floating base robots
\subsection{Robot proprioception}
\subsubsection{What is proprioception?}
\subsubsection{Human and robot proprioception}
\subsubsection{A sensor suite for robot proprioception}
\subsection{State-of-the-art Gradient Descent}
\subsubsection{Fundamentals}
\subsubsection{Momentum gradient descent}
\subsubsection{ADAM gradient descent}
\subsubsection{AMS gradient descent}
\subsection{Fundamentals of graph theory}
\subsubsection{What is a graph?}
\subsubsection{Graph representation, the adjacency matrix}
\subsubsection{Metrics for graph comparison}
\subsection{Network topology inference}
\subsubsection{Detecting linear dependencies with covariance}
\subsubsection{Based on graph signal processing}
\subsubsection{Based on statistic measures}
\subsection{Information theory}
\subsubsection{What is information?}
\subsubsection{The entropy of a random variable}
\subsubsection{Mutual Information: The correlation of the 21st century }
\subsection{Differential geometry}
\subsubsection{Fundamentals of differential geometry}
\subsubsection{Manifolds and the tangent space}
\subsubsection{Riemannian geometry and the metric}
\subsubsection{Use in robotics}

% ===============================================================================================
%                                             |                                                 |
%  --------------------------------------- SECTION ---------------------------------------------|
%                                             |                                                 |
% ===============================================================================================
\section{METHODOLOGY}
\subsection{First-order principles networks}
\subsection{Signal requirements}
\subsubsection{The importance of robot proprioception}
\subsection{Leveraging data history}
\subsubsection{Replay buffer}
\subsubsection{Reservoir sampling}
\subsection{Embodiment and mutual information}
\subsubsection{Online computation of the pairwise mutual information}
\subsubsection{Handling scalar and vector relationships}
\subsubsection{Assessing convergence}
\subsection{Exploratory motions}
\subsubsection{Motor babbling}
\subsubsection{Excitation trajectories}
\subsection{Network topology inference using mutual information}
\subsubsection{Finding the adjacency matrix}
\subsubsection{The proprioceptive information graph}
\subsubsection{Inferring the mechanical topology}
\subsection{Kinematic description guided from topology and proprioceptive measurements}
\subsubsection{Validating bod-body and body-joint connections}
\subsubsection{Finding the sensor-to-sensor orientations}
\subsubsection{Finding the joint center point}
\subsubsection{Learning the kinematics}
\subsection{Online learning inertial parameters}
\subsubsection{Gradient decent for inverse dynamics}
\subsubsection{Constraints for physical feasibility}
%(a) Barrier functions
%(b) Projections
\subsubsection{Riemannian AMS gradient descent}
\subsubsection{The effect of six dimensional force/torque measurements}
\subsubsection{Alternatives to joint acceleration}

% ===============================================================================================
%                                             |                                                 |
%  --------------------------------------- SECTION ---------------------------------------------|
%                                             |                                                 |
% ===============================================================================================
\section{RESULTS}
\subsection{The topology of a robotic arm}
\subsubsection{Contrasting fixed and floating base}
\subsubsection{Comparison with alternative NTI methods}
\subsubsection{Experiment on a physical robot}
\subsection{More complex structures}
\subsubsection{A hexapod robot}
\subsubsection{A humanoid robot}
%(a)  Disambiguation of multiple DoF joints
\subsection{Learning the kinematics}
\subsubsection{Learning of the Denavit-Hartenberg parameters}
\subsubsection{Learning the Euler angles}
\subsubsection{Learning axis orientation and joint center point }
%(a) Arm
%(b) Hexapod
%(c) Humanoid
\subsection{Learning the inertial parameters}
\subsubsection{Learning the inertial parameters with offline optimization}
\subsubsection{Learning the inertial parameters with ADAM gradient decent}
%(a) Handling constraints
\subsubsection{Learning the inertial parameters with Riemannian AMS gradient descent}

% ===============================================================================================
%                                             |                                                 |
%  --------------------------------------- SECTION ---------------------------------------------|
%                                             |                                                 |
% ===============================================================================================
\section{DISCUSSION}
\subsection{Dealing with multiple DoF joints}
\subsection{Analysis of the different sensor types and their potential}
\subsection{The influence of a moving base}
\subsection{The effect of noise}
\subsection{Sample efficiency}
\subsection{Limitations}

% ===============================================================================================
%                                             |                                                 |
%  --------------------------------------- SECTION ---------------------------------------------|
%                                             |                                                 |
% ===============================================================================================
\section{CONCLUSION}
\subsection{Summary}
\subsection{Perspectives}
\subsection{Impact}
\blindtext
	
\end{document}
